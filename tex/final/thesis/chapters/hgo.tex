\chapter[]{Παρατηρητές υψηλού κέρδους}
\label{chap:hgo}
Οι παρατηρητές υψηλού κέρδους είναι ένα εργαλείο το οποίο χρησιμοποιείται στη θεωρία ελέγχου εδώ και τρεις δεκαετίες. Πρωτοεμφανίστηκαν πρώτη φορά στην εργασία~\cite{doyle1979robustness} όπου χρησιμοποιήθηκαν για τη σχεδίαση εύρωστων παρατηρητών για γραμμικά συστήματα, ικανοί να επαναφέρουν τις ιδιότητες ευρωστίας του συστήματος υπό ανάδραση καταστάσεων. Μία δεκαετία αργότερα, άρχισαν να χρησιμοποιούνται σε μη-γραμμικά συστήματα~\cite{esfandiari1987observer,saberi1990observer}. Σε μία εργασία κλειδί~\cite{Esfandiari1992}, οι \textlatin{Khalil} και \textlatin{Esfandiari} έδειξαν πως ήταν πολύ εύκολο ένας τέτοιος παρατηρητής, για αρκούντως μεγάλα κέρδη, να αποσταθεροποιήσει το μη-γραμμικό κλειστό βρόγχο μέσω του \emph{φαινομένου κορύφωσης}. Η λύση που προτάθηκε ήταν απλή: αρκεί ο νόμος ελέγχου να είναι μια ολικώς φραγμένη συνάρτηση των εκτιμήσεων ώστε να υπόκειται κορεσμό κατά τη διάρκεια του έντονου μεταβατικού. Καθώς ο παρατηρητής έχει πολύ μεγάλα κέρδη, είναι πολύ γρηγορότερος από το σύστημα προς έλεγχο, λειτουργώντας ουσιαστικά σε διαφορετική χρονική κλίμακα. Αυτό σημαίνει ότι οι τιμές του συστήματος κατά τη διάρκεια του φαινομένου κορύφωσης παραμένουν πολύ κοντά στις αρχικές τους. Στις εργασίες~\cite{teel1994global,teel1995tools}, οι \textlatin{Teel} και \textlatin{Praly} απέδειξαν για πρώτη φορά την αρχή του διαχωρισμού για μη-γραμμικά συστήματα με τη χρήση παρατηρητών υψηλού κέρδους δίνοντας έτσι ώθηση σε ένα μεγάλο κύμα έρευνας γύρω από τους συγκεκριμένους παρατηρητές και τη χρήση τους για σχεδίαση σχημάτων ελέγχου με ανάδραση εξόδου για την επίλυση προβλημάτων όπως σταθεροποίηση, παρακολούθηση τροχιάς, προσαρμοστικός έλεγχος, κ.α. Σε περίπτωση που ο αναγνώστης θέλει να εμβαθύνει περισσότερο στο αντικείμενο των παρατηρητών υψηλού κέρδους τον παραπέμπουμε στις βιβλιογραφικές αναφορές~\cite{khalil2008high} και~\cite{Khalil2013}.

Για να επιδείξουμε καλύτερα τις ιδιότητες των παρατηρητών υψηλού κέρδους θα αναπαραγάγουμε ένα παράδειγμα με ένα σύστημα δευτέρου βαθμού, παρμένο από την εργασία~\cite{Khalil2013}. Έστω λοιπόν το σύστημα δεύτερης τάξης
\begin{equation}
    \label{eq:hgo_sys1}
    \begin{split}
        \dot x_1 &= x_2 \\
        \dot x_2 &= f(x, u, w, d)\\
        y &= x_1
    \end{split}
\end{equation}
όπου $x = \col\pqty{x_1, x_2}$ είναι το διάνυσμα κατάστασης, $u$ είναι η είσοδος ελέγχου, $y$ είναι η μετρήσιμη έξοδος, $d$ είναι ένα διάνυσμα διαταραχών εισόδου, και $w$ είναι ένα διάνυσμα εξωγενών σημάτων. Η συνάρτηση $f$ είναι τοπικά \lip\ ως προς το $(x,u)$ και συνεχής ως προς το $(w,d)$. Επιπλέον, υποθέτουμε πως οι $w(t)$, $d(t)$, είναι φραγμένες, μετρήσιμες συναρτήσεις του χρόνου. 

Έστω ο νόμος ελέγχου ανάδρασης καταστάσεων $u = \gamma(x,w)$, ο οποίος σταθεροποιεί την αρχή των αξόνων $x = 0$ του συστήματος~\eqref{eq:hgo_sys1}, ομοιόμορφα στο $(w,d)$, με το $\gamma(x,w)$ να είναι τοπικά \lip\ στο $x$ και συνεχές στο $w$. Για να υλοποιήσουμε αυτόν τον νόμο ελέγχου χρησιμοποιώντας μόνο μετρήσεις της εξόδου $y$, χρησιμοποιούμε τον ακόλουθο παρατηρητή:
\begin{equation}
    \label{eq:hgo observer structure}          
    \begin{split}
        \dot{\hat{x}}_1 &= \hat x_2 + h_1 \pqty{y - \hat x_1}\\
        \dot{\hat{x}}_2 &= \hat f(\hat{x}, u, w) + h_2(y -\hat x_1)
    \end{split}
\end{equation}
όπου $\hat f(\hat{x}, u, w)$ είναι ένα μοντέλο της $f(w, u, w, d)$. Αν η $f$ είναι μία γνωστή συνάρτηση των $(x,u, w)$, μπορούμε να επιλέξουμε $\hat f = f$. Εναλλακτικά, αν δεν έχουμε κάποια γνώση για τις μη-γραμμικότητες του συστήματος, μπορούμε να επιλέξουμε $\hat f =0$. Tο σφάλμα εκτίμησης
\[
    \tilde x = \bmqty{\tilde{x}_1\\\tilde{x}_2}
             = \bmqty{x_1 - \hat x_1\\ x_2 - \hat x_2}
\]
θα έχει την ακόλουθη δυναμική:
\begin{equation}
    \label{eq:hgo estimation error dynamics}
    \begin{split}
        \dot{\tilde{x}}_1 &= -h_1 \tilde{x}_1 + \tilde x_2\\
        \dot{\tilde{x}}_2 &= -h_2 \tilde{x}_2 + \delta(x, \tilde x, w, d)
    \end{split}
\end{equation}
όπου
\[
    \delta(x, \tilde x, w, d) = f(x, \gamma(\hat x, w), w, d) - \hat f(\hat x, \gamma(\hat x, w), w).
\]
Εν απουσία του $\delta$, μπορούμε να πετύχουμε ασυμπτωτική σύγκλιση των σφαλμάτων εκτίμησης στο μηδέν, αρκεί ο πίνακας
\[
   A_0 = \bmqty{-h_1 & 1 \\ -h_2 &0}
\]
να είναι \hur, που ισχύει για οποιεσδήποτε θετικές σταθερές $h_1$, $h_2$. Παρουσία του $\delta$, πρέπει να σχεδιάσουμε τον $A_0$ με την επιπλέον ιδιότητα να απορρίπτει την διαταραχή του $\delta$ στο $\tilde x$. Αυτό μπορεί ιδανικά να επιτευχθεί αν η συνάρτηση μεταφοράς από το $\delta$ στο $\tilde x$
\[
    G_o(s) = \frac{1}{s^2 + h_1 s + h_2} \bmqty{1 \\ s + h_1}
\]
ταυτίζεται με το μηδέν. Αυτό για να επιτευχθεί θα απαιτούσε τα κέρδη να είναι άπειρα, αλλά πρακτικά μπορούμε να κάνουμε το $\sup_{\omega \in \R} \norm{G_o(j \omega)}$ αυθαίρετα μικρό, επιλέγοντας $h_2 \gg h_1 \gg 1$. Συνεπώς, θεωρώντας
\[
    h_1 = \frac{\eta_1}{\mu},\quad h_2 = \frac{\eta_2}{\mu^2}
\]
για κάποιες θετικές σταθερές $\eta_1$, $\eta_2$, και $\mu$, όπου $\mu \ll 1$, η συνάρτηση μεταφοράς $G_o(s)$ πλέον μπορεί να γραφεί ως
\[
    G_o(s) = \frac{\mu}{(\mu s)^2 + \eta_1 \mu s + \eta_2} \bmqty{1\mu \\ \mu s + \eta_1}.
\]
Είναι προφανές πως $\lim_{\mu \rightarrow 0} G_o(s) = 0$. Επιλέγοντας τα $h_1$, $h_2$ με αυτόν τον τρόπο, οι ιδιοτιμές του παρατηρητή είναι $\flatfrac{1}{\mu}$ φορές οι ρίζες του πολυωνύμου $s^2 + \eta_1 s + \eta_2$. Έτσι, όταν το $\mu$ είναι μικρό, τότε η δυναμική του παρατηρητή θα είναι πολύ πιο γρήγορη συγκριτικά με το σύστημα του κλειστού βρόγχου με ανάδραση καταστάσεων~\eqref{eq:hgo_sys1}.

H απόρριψη διαταραχών ενός παρατηρητή υψηλού κέρδους μπορεί να δειχθεί και με ανάλυση στο πεδίο του χρόνου, χρησιμοποιώντας τα κανονικοποιημένα σφάλματα εκτίμησης
\[
    \zeta_1 = \frac{\tilde{x}_1}{\mu},\quad \zeta_2 = \tilde x_2,
\]
τα οποία υπακούν στην ακόλουθη δυναμική:
\begin{equation}
    \label{eq:hgo scaled estimation error dynamics}          
    \begin{split}
        \mu \dot \zeta_1 & = - \eta_1 \zeta_1 + \zeta_2\\
        \mu \dot \zeta_2 & = - \eta_2 \zeta_2 + \mu \delta(x, \tilde x, w, d)
    \end{split}
\end{equation}
Είναι εμφανές από το άνω σύστημα ότι η μείωση του $\mu$ οδηγεί σε μικρότερη επίδραση του $\delta$. Στην μεταβατική απόκριση του παρατηρητή υψηλού κέρδους εμφανίζεται το φαινόμενο κορύφωσης, γνωστό και ως \emph{\textlatin{peaking phenomenon}} στην ξενόγλωσση βιβλιογραφία. Η αρχική συνθήκη $\zeta_1(0)$ μπορεί να είναι της τάξης του $\bigo{\flatfrac{1}{\mu}}$ όταν $x_1(0) \neq \hat x_1(0)$. Επομένως, η μεταβατική απόκριση του~\eqref{eq:hgo scaled estimation error dynamics} θα περιέχει έναν όρο της μορφής $(\flatfrac{1}{\mu})\exp(- \flatfrac{\alpha t}{\mu})$ για κάποιο $\alpha > 0$. Παρόλο που αυτός ο εκθετικός όρος αποσβένει ταχύτατα, εμφανίζει μία σχεδόν-κρουστική συμπεριφορά όπου κατά το μεταβατικό φαινόμενο φτάνει σε τιμές της τάξης του $\bigo{\flatfrac{1}{\mu}}$ πριν μειωθεί αστραπιαία προς το μηδέν. Μάλιστα, η συνάρτηση $(\flatfrac{1}{\mu})\exp(- \flatfrac{\alpha t}{\mu})$ προσεγγίζει την κρουστική συνάρτηση καθώς το $\mu$ τείνει στο μηδέν. Επιπροσθέτως, πέρα από μη-αποδεκτή μεταβατική συμπεριφορά, το φαινόμενο αυτό μπορεί εντέλει να οδηγήσει τον κλειστό βρόγχο σε αστάθεια~\cite{Esfandiari1992}.

Για να μελετήσουμε πως μπορεί να συμβεί αυτό, θα επικαλεστούμε το Θεώρημα 3 της εργασίας~\cite{Esfandiari1992}, το οποίο δηλώνει πως δεδομένου του συστήματος \cref{eq:hgo_sys1} και του νόμου ελέγχου $u = \gamma(x,w)$, o οποίος μετατρέπει την αρχή των αξόνων του \cref{eq:hgo_sys1} σε ένα (τοπικά) ασυμπτωτικά ευσταθές σημείο ισορροπίας, και ενός παρατηρητή \cref{eq:hgo observer structure} με κατάλληλα επιλεγμένα κέρδη $h_1 = \flatfrac{\eta_1}{\mu}$, $h_2 = \flatfrac{\eta_2}{\mu^2}$, υπάρχει ένα $\bar \mu > 0$, τέτοιο ώστε για κάθε $\mu \in (0, \bar \mu)$, η αρχή των αξόνων του επαυξημένου συστήματος \cref{eq:hgo_sys1,eq:hgo scaled estimation error dynamics} με νόμο ελέγχου $u = \gamma(\hat x, w)$ θα παραμείνει ένα (τοπικά) ευσταθές σημείο ισορροπίας. Για κάθε αρχική κατάσταση που βρίσκεται εντός της της περιοχής έλξης της αρχής των αξόνων, οι τροχιές του συστήματος υπό ανάδραση εξόδου θα προσεγγίζουν τις ιδανικές τροχιές που εμφανίζονται υπό ανάδραση καταστάσεων καθώς το $\mu \rightarrow 0$. Όμως, καθώς το $\mu$ τείνει προς το μηδέν, η περιοχή έλξης συρρικνώνεται, πράγμα το οποίο σημαίνει ότι οι αρχικές συνθήκες μπορούν να βρεθούν εκτός αυτής, προκαλώντας έτσι την αστάθεια του κλειστού βρόγχου. Στην περίπτωση που η αρχή των αξόνων είναι ολικά ασυμπτωτικά ευσταθής, δηλαδή η περιοχή έλξης είναι όλο το $\R^2$, δεν τίθεται θέμα συρρίκνωσής της, καθώς ο χώρος $\R$ είναι αμετάβλητος ως προς την κλιμάκωση.

Ένας τρόπος αντιμετώπισης του φαινομένου αυτού είναι να σχεδιαστεί ο νόμος ελέγχου $\gamma(\hat x, w)$ και η συνάρτηση $\hat f(\hat x, u, w)$ ώστε να είναι ολικώς φραγμένα στο $\hat x$, δηλαδή να είναι φραγμένα για κάθε $\hat x$ όταν το $w$ είναι φραγμένο. Αυτό μπορεί εύκολα να επιτευχθεί προσθέτοντας στο $u$ ή/και το $\hat x$ συναρτήσεις κορεσμού, σχεδιασμένες έτσι ώστε τα σημεία κορεσμού να βρίσκονται εκτός των επιθυμητών συνόλων λειτουργίας. Το ολικό φραγμένο του $\gamma$ και του $\hat f$ ως προς το $\hat x$ προστατεύει το σύστημα από το φαινόμενο κορύφωσης, καθώς κατά τη διάρκεια αυτού, η είσοδος ελέγχου $\gamma(\hat x, w)$ φτάνει στον κορεσμό. Καθώς η διάρκεια του φαινομένου τείνει στο μηδέν καθώς το $\mu$ μικραίνει, για επαρκώς μικρό $\mu$ η διάρκειά του είναι τόσο μικρή ώστε η κατάσταση του συστήματος $x$ παραμένει κοντά στην αρχική της τιμή. Με το πέρας του φαινομένου, το σφάλμα εκτίμησης είναι πλέον της τάξης του $\bigo{\mu}$ και ο νόμος ελέγχου $\gamma(\hat x, w)$ είναι $\bigo{\mu}$ κοντά στο $\gamma(x, w)$. Συνεπώς, οι τροχιές του συστήματος κλειστού βρόγχου υπό έλεγχο ανάδρασης εξόδου προσεγγίζουν τις τροχιές του υπό πλήρη ανάδραση καταστάσεων καθώς το $\mu$ τείνει στο μηδέν.

Ο κλειστός βρόγχος υπό ανάδραση εξόδου μπορεί να γραφεί λοιπόν ως εξής:
\begin{subequations}
    \label{eq:hgo singular pertubation form}          
    \begin{align}
        \label{eq:hgo singular pertubation form slow a}
        \dot x_1 & = x_2 \\
        \label{eq:hgo singular pertubation form slow b}
        \dot x_2 & = f(x, \gamma(\hat x, w), w, d)\\
        \label{eq:hgo singular pertubation form fast a}
        \mu \dot \zeta_1 & = - \eta_1 \zeta_1 + \zeta_2\\
        \label{eq:hgo singular pertubation form fast b}
        \mu \dot \zeta_2 & = - \eta_2 \zeta_2 + \mu \delta(x, \tilde x, w, d)
    \end{align}
\end{subequations}
όπου $\hat x_1 = x_1 - \mu \zeta_1$ kai $\hat x_2 = x_2 - \zeta_2$. Το «αργό» δυναμικό σύστημα~\eqref{eq:hgo singular pertubation form slow a}-\eqref{eq:hgo singular pertubation form slow b} ταυτίζεται με τον κλειστό βρόγχο υπό ανάδραση καταστάσεων όταν $\zeta = 0$. To ομογενές μέρος της «γρήγορης» διαφορικής εξίσωσης ~\eqref{eq:hgo singular pertubation form fast a}-\eqref{eq:hgo singular pertubation form fast b} είναι $\mu \dot \zeta = A_0 \zeta$. Θεωρούμε τη συνάρτηση \lyap\ $V(x)$ για το αργό υποσύστημα, της οποίας η ύπαρξη είναι εγγυημένη δεδομένου ενός νόμου ελέγχου που σταθεροποιεί το σύστημα. Θεωρούμε επιπλέον τη συνάρτηση \lyap\ για το γρήγορο υποσύστημα $W(\zeta) = \zeta^\T P_0 \zeta$, όπου $P_0 = P_0^\T >0$ είναι η λύση της εξίσωσης $P_0 A_0 + A_0^\T P_0 = - I$. Ορίζουμε τα σύνολα
\[
    \Omega_c = \qty{x \in \R^2: V(x) \leq c} \qq{και}
    \Sigma = \qty{\zeta \in \R^2: W(\zeta) \leq \varrho \mu^2},
\]
όπου για κάθε $c >0$, το $\Omega_c$ βρίσκεται στο εσωτερικό της περιοχής έλξης του~\eqref{eq:hgo_sys1}. Η ανάλυση χωρίζεται σε δύο βήματα: αρχικά, δείχνουμε ότι για κατάλληλα επιλεγμένο $\varrho$, υπάρχει ένα $\mu_1^*>0$ τέτοιο ώστε, για κάθε $0 < \mu < \mu_1^*$, η αρχή των αξόνων του κλειστού βρόγχου είναι ασυμπτωτικά ευσταθής, και το σύνολο $\Omega_c \times \Sigma$ είναι ένα θετικώς αμετάβλητο υποσύνολο της περιοχής έλξης. Η απόδειξη χρησιμοποιεί το γεγονός ότι το $\zeta$ είναι της τάξης $\bigo \mu$ στο $\Omega_c \times \Sigma$. Στο δεύτερο βήμα, δείχνουμε ότι για κάθε φραγμένο $\hat x(0)$ και για κάθε $x(0) \in \Omega_b$, όπου $\Omega_b = \qty{x \in \R^2: V(x) \leq b}$, με $0 < b < c$, υπάρχει ένα $\mu_2^*>0$ τέτοιο ώστε, για κάθε $0 < \mu < \mu_2^*$, η τροχιά εισέρχεται στο εσωτερικό του $\Omega_c\times \Sigma$ σε πεπερασμένο χρόνο. H απόδειξη χρησιμοποιεί το γεγονός ότι το $\Omega_b$ είναι γνήσιο υποσύνολο του $\Omega_c$ και το $\gamma(\hat x, w)$ είναι ολικώς φραγμένο. Υπάρχει χρόνος $\tau_1 > 0$, ανεξάρτητος του $\mu$, τέτοιος ώστε κάθε τροχιά που ξεκινάει από το $\Omega_b$ να παραμένει στο $\Omega_c$ για κάθε $t \in [0, \tau_1]$. Δεδομένου ότι το $\zeta$ αποσβένει γρηγορότερα από ένα εκθετικό της μορφής $(\flatfrac{1}{\mu}) \exp(\flatfrac{-\alpha t}{\mu})$, μπορούμε να δείξουμε ότι η τροχιά εισέρχεται του σύνολου $\Omega_c \times \Sigma$ εντός του χρονικού διαστήματος $[0, \tau(\mu)]$, όπου $\lim_{\mu \rightarrow 0}\tau(\mu) =0$. Επομένως, με την επιλογή επαρκώς μικρού $\mu$, μπορούμε να εγγυηθούμε ότι $\tau(\mu) < \tau_1$. Επιπλέον, εξαιτίας του ολικού φραγμένου της δεξιάς πλευράς της~\eqref{eq:hgo singular pertubation form slow a}--\eqref{eq:hgo singular pertubation form slow b}, επιλέγοντας επαρκώς μικρό $\mu$, μπορούμε να κάνουμε τη διαφορά $\norm{x(\tau(\mu)) - x(0)}$ οσοδήποτε μικρή. Συνδυάζοντας αυτό με το γεγονός ότι $\zeta(t) \in \bigo \mu$, όταν $t \geq \tau(\mu)$, εύκολα αποδεικνύεται ότι οι τροχιές $x(t)$ υπό ανάδραση εξόδου μπορούν να είναι αυθαίρετα κοντά σε αυτές υπό ανάδραση καταστάσεων για κάθε $t \geq 0$.

Η σχεδίαση ενός ελεγκτή ανάδρασης εξόδου μπορεί να αναχθεί σε μία διαδικασία που χωρίζεται σε δύο μέρη: αρχικά σχεδιάζεται ένας ελεγκτής ανάδρασης καταστάσεων που να επιτυγχάνει τους επιθυμητούς στόχους ελέγχου, θεωρώντας πως όλες οι καταστάσεις είναι διαθέσιμες προς μέτρηση; έπειτα σχεδιάζεται ένας παρατηρητής - ανεξάρτητος από τον έλεγχο ανάδρασης καταστάσεων - με σκοπό την ανακατασκευή των μη-διαθέσιμων καταστάσεων από μετρήσεις της εξόδου. Επιλέγοντας αρκετά μικρό $\mu$, ο ελεγκτής ανάδρασης εξόδου μπορεί να πετύχει τα χαρακτηριστικά ευστάθειας και επίδοσης του ελεγκτή ανάδρασης καταστάσεων. Αυτή η αρχή του διαχωρισμού εμφανίζεται και στα γραμμικά συστήματα, όπου οι ιδιοτιμές του κλειστού βρόγχου με τη χρήση ενός σχήματος παρατηρητή-ελεγκτή αποτελούνται από την ένωση των ιδιοτιμών υπό ανάδραση καταστάσεων και του παρατηρητή. Συνεπώς, η σταθεροποίηση του κλειστού βρόγχου μπορεί να επιτευχθεί λύνοντας δύο διαφορετικά προβλήματα τοποθέτησης ιδιοτιμών: για την ανάδραση καταστάσεων και για τον παρατηρητή. Αξίζει να σημειωθεί ότι η επίτευξη παρόμοιων αποτελεσμάτων σε μη-γραμμικά συστήματα αποτελεί τομέα με έντονο ακαδημαϊκό ενδιαφέρον. Οι παρατηρητές υψηλού κέρδους έχουν μία μοναδική ιδιότητα που δεν εμφανίζεται σε άλλα αποτελέσματα σχετικά με την αρχή του διαχωρισμού (συμπεριλαμβανομένων και των γραμμικών συστημάτων): κάνοντας τον παρατηρητή αρκετά γρήγορο μπορούμε να προσεγγίσουμε τις τροχιές του κλειστού βρόγχου υπό ανάδραση καταστάσεων. Μπορούμε να εκμεταλλευτούμε αυτήν την ιδιότητα, σχεδιάζοντας έναν ελεγκτή ανάδρασης καταστάσεων που να πετυχαίνει τις επιθυμητές προδιαγραφές απόκρισης και έπειτα εφαρμόζοντας μία συνάρτηση κορεσμού στις εκτιμήσεις $\hat x$ ή/και στην είσοδο ελέγχου $u$ εκτός των επιθυμητών χωρίων λειτουργίας. Έτσι, οι συναρτήσεις 
$\gamma(\hat x, w)$ και $\hat f(\hat x, u, w)$ είναι ολικώς φραγμένες στο $\hat x$ και έτσι μπορούμε με κατάλληλη ρύθμιση του $\mu$ να κάνουμε τις τροχιές υπό ανάδραση εξόδου να εμφανίζουν παρόμοια συμπεριφορά με αυτές υπό ανάδραση καταστάσεων.

Για να δείξουμε αυτή την ιδιότητα, ας θεωρήσουμε την ακόλουθη αλληλουχία ολοκληρωτών:
\begin{align*}
    \dot x_1 &= x_2\\
    \dot x_2 &= u
\end{align*}
που αποτελεί μια γραμμική έκφανση του~\eqref{eq:hgo_sys1} με $f = u$. Έστω και ο γραμμικός νόμος ανάδρασης καταστάσεων
\[
    u = -2x_1 -2 x_2
\]
που τοποθετεί τις ιδιοτιμές στο $1 \pm j$. Επιπλέον, ο παρατηρητης
\begin{align*}
    \dot {\hat x}_1 &= \hat x_2 + (\flatfrac{3}{\mu}) (y - \hat x_1)\\
    \dot {\hat x}_2 &= u +  (\flatfrac{2}{\mu^2}) (y - \hat x_1)
\end{align*}
είναι μια ειδική περίπτωση του~\eqref{eq:hgo observer structure} και οι ιδιοτιμές του βρίσκονται στο $\flatfrac{-1}{\mu}$ και στο $\flatfrac{-2}{\mu}$. Συνεπώς, ο ελεγκτής ανάδρασης εξόδου 
\begin{equation}
    \label{eq:hgo ofb control w/o sat}
    u = u(\hat x) = - 2 \hat{x}_1 - 2 \hat x_2,
\end{equation}
που κάνει χρήση του παρατηρητή, τοποθετεί τις ιδιοτιμές του κλειστού βρόγχου στα $1 \pm j$, $\flatfrac{-1}{\mu}$, και $\flatfrac{-2}{\mu}$, και συνεπώς για κάθε $\mu > 0$, η αρχή των αξόνων είναι ασυμπτωτικά ευσταθής για το σύστημα κλειστού βρόγχου. Στο \cref{fig:hgo_sfb_ofb_mu_comparison_no_sat} παρουσιάζονται τα αποτελέσματα μίας προσομοίωσης του συστήματος υπό συζήτηση με $x_1(0) = 1$ και $x_2(0) = \hat x_1(0) = \hat x_2(0) = 0$, χωρίς όμως να εφαρμοστεί κάποιος κορεσμός στον ελεγκτή ανάδρασης εξόδου. Παρατηρεί λοιπόν κανείς, ότι ενώ το $\mu$ μειώνεται, οι τροχιές υπό ανάδραση εξόδου δεν πλησιάζουν αυτές που εμφανίζονται υπό ανάδραση καταστάσεων. Αυτή η συμπεριφορά αιτιολογείται από την έλλειψη κορεσμού που επιτρέπει το έντονο μεταβατικό να μεταφερθεί στο σύστημα υπό έλεγχο (\textlatin{peaking}). Στο \cref{fig:hgo_sfb_ofb_mu_comparison_sat} επαναλαμβάνουμε την ίδια προσομοίωση, όμως αυτή τη φορά η είσοδος ελέγχου υπόκειται κορεσμό στο $\pm 4$, δηλαδή πλέον έχουμε 
\begin{equation}
    \label{eq:hgo ofb control w/ sat}
    u = u_s(\hat x) = 4 \sat\pqty{\flatfrac{\pqty{- 2 \hat{x}_1 - 2 \hat x_2}}{4}}.
\end{equation}
Το συγκεκριμένο επίπεδο κορεσμού επιλέχθηκε ώστε να ικανοποιείται η σχέση $4 > \max_{x \in\Omega} \abs{-2x_1 -2x_2}$ με το 
\[
    \Omega = \qty{x \in \R^2: 1.25 x_1^2 + 0.5 x_1 x_2 + 0.375x_2^2 \leq 1.4}
\]
να αποτελεί μια εκτίμηση της περιοχής έλξης υπό ανάδραση καταστάσεων που περιλαμβάνει την αρχική κατάσταση $(1,0)$ στο εσωτερικό της. Είναι λοιπόν εμφανές πως στο \cref{fig:hgo_sfb_ofb_mu_comparison_sat}, τα πειραματικά αποτελέσματα συμπίπτουν με την ανάλυση, καθώς μειώνοντας το $\mu$ οι $x$-τροχιές υπό ανάδραση εξόδου πλησιάζουν αυτές που εμφανίζονται υπό ανάδραση καταστάσεων. Στο \cref{fig:hgo_u_comparison} αντιπαραθέτουμε τους ελεγκτές~\eqref{eq:hgo ofb control w/o sat} και~\eqref{eq:hgo ofb control w/ sat}, όπου η σημασία του κορεσμού γίνεται εμφανής και γραφικά. Τέλος, αξίζει να αναφερθεί πως πολύ υψηλά επίπεδα κορεσμού μπορούν να προκαλέσουν παρόμοια συμπεριφορά με την αύξηση του $\mu$, όπως φαίνεται και στο \cref{fig:hgo_sfb_ofb_sat_comparison}; αυτό βέβαια είναι απολύτως λογικό καθώς ένα πιο συντηρητικό σημείο κορεσμού επιτρέπει περισσότερη ενέργεια από το έντονο μεταβατικό του παρατηρητή να μεταφερθεί στο σύστημα. Παρόλα αυτά όμως για οποιοδήποτε επίπεδο κορεσμού, θα υπάρχει πάντα μία κατάλληλη σταθερά κλιμάκωσης χρόνου $\mu$ που να καθιστά τον τελικό κλειστό βρόχο ευσταθή.

\begin{figure}[t]
    \selectlanguage{english}
    \centering
    % This file was created by matlab2tikz.
%
\tikzstyle{every node}=[font=\footnotesize]
\begin{tikzpicture}

\begin{axis}[%
width=\figurewidth,
height=0.465\figureheight,
at={(0\figurewidth,0.507\figureheight)},
scale only axis,
xmin=0,
xmax=10,
ymin=-0.32,
ymax=1.05,
xticklabels = \empty,
ylabel style={font=\color{white!15!black}},
ylabel={$x_1(t)$},
axis background/.style={fill=white},
legend style={legend cell align=left, align=left, draw=white!15!black},
xlabel style={font=\footnotesize},ylabel style={font=\footnotesize},ticklabel style={font=\scriptsize},x tick label style={font=\scriptsize},y tick label style={font=\scriptsize},legend style={font=\scriptsize}
]
\addplot [color=black]
  table[row sep=crcr]{%
0	1\\
0.0218942716257455	0.99952759927239\\
0.0427146117533166	0.998226863318076\\
0.0606493011849629	0.996468133364459\\
0.0831047969961443	0.993468283471582\\
0.105560292807326	0.989620515567882\\
0.128015788618507	0.9849658623348\\
0.157523732279078	0.977689649632749\\
0.19408412378904	0.966969351042962\\
0.230644515299	0.954512736001034\\
0.267204906808962	0.940474339768624\\
0.307423722456617	0.923380182721619\\
0.347642538104273	0.904737717184577\\
0.387861353751928	0.884724540991085\\
0.429537038275953	0.862719757185671\\
0.472669591676345	0.838773929160828\\
0.515802145076737	0.813818013151536\\
0.581814200366589	0.774057918857673\\
0.650452706034962	0.731241120674108\\
0.74320522717408	0.671983143368255\\
0.938712875370294	0.546654491145329\\
1.0149482084279	0.499096449070572\\
1.09269747268888	0.451986613503758\\
1.14654889046735	0.420353507593624\\
1.20040030824582	0.389640108853536\\
1.25425172602429	0.359921307038547\\
1.31180609730742	0.329328124423034\\
1.36936046859054	0.300001586732961\\
1.42691483987367	0.271987392901766\\
1.4844692111568	0.24531806491693\\
1.54202358243992	0.220014177212095\\
1.59957795372305	0.19608552701839\\
1.65713232500617	0.173532244916727\\
1.7146866962893	0.15234584508018\\
1.76242675867043	0.135798010425626\\
1.81016682105157	0.120166939268685\\
1.85790688343271	0.10543707623491\\
1.90564694581385	0.0915903661244819\\
1.95338700819499	0.0786065599188941\\
2.00112707057612	0.0664635026285918\\
2.04886713295726	0.0551374030972571\\
2.0966071953384	0.0446030864009579\\
2.15985263984993	0.0318215829913306\\
2.21891575983136	0.0210421612577836\\
2.27797887981279	0.0113253517264962\\
2.34329566259284	0.00175134509642838\\
2.40865963699659	-0.00666989983725585\\
2.47416518627146	-0.0140205532699156\\
2.54327833284418	-0.0206767956654055\\
2.61599907671476	-0.0265552970243697\\
2.69457411194424	-0.031722022292314\\
2.77510057762669	-0.0358627727900309\\
2.86141739359006	-0.0391439684288422\\
2.94773420955343	-0.0413698613504767\\
3.04803397601946	-0.0428115143537138\\
3.15472043949996	-0.0432065373397279\\
3.2791879802272	-0.0424682433717738\\
3.42430975940991	-0.0403651000211198\\
3.58735440541699	-0.0368978699432176\\
3.83201022720331	-0.0305022725955926\\
4.36206800044872	-0.0163539401425457\\
4.61990719697795	-0.0107215914643959\\
4.85088062717586	-0.00666684692260944\\
5.1107257361485	-0.00321985305828143\\
5.37057084512115	-0.000834595695170748\\
5.65928763286853	0.000792505208915273\\
6.00574777816539	0.00169528333523239\\
6.46967388218016	0.00181019578260155\\
7.60669008461243	0.000603674828244749\\
8.81857555088457	-3.72956661269797e-05\\
10	-6.27923251883544e-05\\
};
\addlegendentry{SFB}

\addplot [color=black, dashed]
  table[row sep=crcr]{%
0	1\\
0.019360369371805	0.999515537057546\\
0.0314175227428173	0.998095695381247\\
0.0415890390134344	0.995877355284753\\
0.0515267024389114	0.992661769060053\\
0.0608404118483552	0.98863404368759\\
0.0708210363215542	0.983186783785886\\
0.0812842910065452	0.976204122581478\\
0.0921085387076541	0.967621115774655\\
0.103290696633616	0.957338970407093\\
0.114825124338461	0.945286297505531\\
0.126773454229511	0.931335079387583\\
0.139060262697381	0.915535591366325\\
0.153803349629255	0.894793718500635\\
0.16899392536155	0.871591519257331\\
0.186851026214196	0.842240833437456\\
0.205185430773707	0.810124556413699\\
0.226262458583786	0.771189840908944\\
0.252186777018208	0.721063267926647\\
0.287379419659059	0.650475127802554\\
0.401706265906038	0.419129254673663\\
0.435279706426078	0.354514737189001\\
0.467494035834292	0.295022901525224\\
0.492229488838138	0.251235456163705\\
0.517837934459795	0.207773928153905\\
0.544319372699263	0.164931916821045\\
0.571729788586421	0.12291108767889\\
0.599449863689474	0.0828654991723887\\
0.620239920016763	0.0544596490122409\\
0.64668712085501	0.0203385202271207\\
0.669315698898835	-0.00707923492362283\\
0.692363420860573	-0.0333428309491079\\
0.715830286740221	-0.0583905407926633\\
0.736225504817101	-0.0788001119534592\\
0.758018257227553	-0.0992434213341902\\
0.780738829896167	-0.119094618135383\\
0.805315043081103	-0.138937234825478\\
0.829891256266041	-0.157144212378999\\
0.854921906889258	-0.174070818613812\\
0.879952557512475	-0.189432648686283\\
0.905344439584102	-0.20348783217835\\
0.930916937379937	-0.216160073295628\\
0.956647330794269	-0.22748305369568\\
0.982693515445602	-0.237562898799963\\
1.00873970009693	-0.24632848575262\\
1.03516819152223	-0.253956717449194\\
1.06159668294753	-0.260386279967058\\
1.08820575132635	-0.265724702902979\\
1.11490510818194	-0.270011066820002\\
1.14160446503752	-0.273297946929549\\
1.16830382189311	-0.275655343337048\\
1.1950031787487	-0.277150576345285\\
1.22579014860531	-0.277888583500689\\
1.25718875946968	-0.277646715693358\\
1.29153332669318	-0.276343244259877\\
1.33220183953005	-0.273551556546325\\
1.37287035236691	-0.269577106839789\\
1.41511294279164	-0.264374666395538\\
1.46440278199243	-0.257144488949749\\
1.5135337862116	-0.24892514559061\\
1.56940238230018	-0.238623744556302\\
1.64218137027166	-0.224117003922043\\
1.747976156312	-0.201800422571845\\
1.94337384944909	-0.160293778130374\\
2.04042788417053	-0.140780412874939\\
2.12687370071784	-0.124429616513519\\
2.20062205673976	-0.111367011437714\\
2.27437041276167	-0.0991794140497415\\
2.34811876878359	-0.0878934162701288\\
2.42186712480551	-0.077513581516266\\
2.49561548082742	-0.0680271751080799\\
2.56936383684934	-0.059408180591717\\
2.66154928187673	-0.0497988094915272\\
2.75373472690413	-0.0414030818369984\\
2.84592017193152	-0.0341255190171541\\
2.93810561695892	-0.0278656136389888\\
3.04872815099179	-0.0215543273217467\\
3.15935068502467	-0.0163916235783947\\
3.26997321905754	-0.0122160529081903\\
3.39903284209589	-0.00839448271478993\\
3.54652955413973	-0.00515508632601325\\
3.71246335518904	-0.00262866680494867\\
3.91527133424931	-0.000710694947303026\\
4.15746063139286	0.000461438023341643\\
4.49857690837231	0.000965654667215432\\
5.17911965261986	0.000635045007401303\\
6.66021136563755	3.56870090261197e-05\\
10	-1.20033652351026e-07\\
};
\addlegendentry{$\text{OFB w/o sat, }\mu = 0.1$}

\addplot [color=black, dotted]
  table[row sep=crcr]{%
0	1\\
0.00472125223863173	0.999492078224977\\
0.008299324578811	0.997823149105759\\
0.0120988754146634	0.994668016486916\\
0.016514657073877	0.98947298968981\\
0.0223820221100208	0.980714883745216\\
0.0307515534799911	0.966021345496264\\
0.0452578292723036	0.938009944021422\\
0.100997533340982	0.829024088922143\\
0.132904862131745	0.769431765410326\\
0.16398158131036	0.713534309447367\\
0.194285161440199	0.661043880614013\\
0.22638184428002	0.607591628873067\\
0.256007655703179	0.560185168808529\\
0.284077784585254	0.516953537839404\\
0.315934114123497	0.469846534016492\\
0.340772911441222	0.434536698445656\\
0.365611708758948	0.400451279881471\\
0.396552642595084	0.359676091381086\\
0.427493576431221	0.320733880441816\\
0.458434510267358	0.283585827540691\\
0.494915334794298	0.242036985144068\\
0.515925692403169	0.219188521443625\\
0.53693605001204	0.197113799308557\\
0.557946407620911	0.17579967807966\\
0.578956765229782	0.155232896418577\\
0.599967122838653	0.135400089276088\\
0.620977480447525	0.116287804215665\\
0.659418719992781	0.0831397509717711\\
0.694280483865553	0.0550549137352565\\
0.729142247738324	0.0287856823159096\\
0.75766967322728	0.00859664095130164\\
0.786197098716238	-0.0104526133877716\\
0.814724524205195	-0.0283961556153631\\
0.844557718062582	-0.0460150768903951\\
0.875696680288398	-0.0631966860578537\\
0.906835642514213	-0.0791868693779616\\
0.937974604740027	-0.0940287676966722\\
0.974487035238701	-0.110026697471397\\
1.01099946573737	-0.124572408133542\\
1.04751189623605	-0.137732724674283\\
1.0858807257019	-0.150141229314643\\
1.12610595413494	-0.161666697499903\\
1.16633118256798	-0.171756510452948\\
1.20655641100102	-0.180492667036464\\
1.24799582774988	-0.188161236641333\\
1.28943524449874	-0.194562608418673\\
1.3308746612476	-0.199778870783975\\
1.37231407799646	-0.203889410492298\\
1.41375349474533	-0.206970864427968\\
1.46822526883875	-0.209579724496347\\
1.51584206783315	-0.210634045699466\\
1.57130100093479	-0.210561579988397\\
1.62844233666214	-0.209176465052122\\
1.68726607501519	-0.206524930623159\\
1.73770192110293	-0.203381821364777\\
1.80916174410189	-0.197746250613601\\
1.88650079468855	-0.190343249776095\\
1.96579958780444	-0.181635943695108\\
2.07247144193488	-0.168614695042795\\
2.20582332736357	-0.151024291587028\\
2.59379272526447	-0.0989158174345235\\
2.72746493551845	-0.0824453423608666\\
2.83609718298797	-0.070023758375088\\
2.94571418218135	-0.0584721204990739\\
3.05533118137473	-0.0479633443076306\\
3.16494818056811	-0.0385191107160008\\
3.27456517976149	-0.0301331365001172\\
3.39762907901465	-0.0219434271383818\\
3.5113880784421	-0.0154705967363142\\
3.62514707786955	-0.00998423632318612\\
3.738906077297	-0.0054083129176643\\
3.87541687660995	-0.00100366144149611\\
4.01192767592289	0.00235350106936316\\
4.17119027512132	0.00513935738218407\\
4.33045287431974	0.00691681798250521\\
4.51246727340366	0.00797342811543089\\
4.73998527225856	0.00822138263952255\\
5.03887098102951	0.00740684911144385\\
5.62040673991143	0.00446068191955895\\
6.23562898412938	0.00174304287210525\\
6.80030545081482	0.00033491442493272\\
7.49978375042503	-0.000279278781535552\\
8.90506426484099	-0.000146190864217743\\
10	-9.5820952985548e-06\\
};
\addlegendentry{$\text{OFB w/o sat, }\mu = 0.01$}

\end{axis}

\begin{axis}[%
width=\figurewidth,
height=0.465\figureheight,
at={(0\figurewidth,0\figureheight)},
scale only axis,
xmin=0,
xmax=10,
xlabel style={font=\color{white!15!black}},
xlabel={Time, $t$ [\si{\second}]},
ymin=-2.1,
ymax=0.3,
ylabel style={font=\color{white!15!black}},
ylabel={$x_2(t)$},
axis background/.style={fill=white},
xlabel style={font=\footnotesize},ylabel style={font=\footnotesize},ticklabel style={font=\scriptsize},x tick label style={font=\scriptsize},y tick label style={font=\scriptsize},legend style={font=\scriptsize}
]
\addplot [color=black, forget plot]
  table[row sep=crcr]{%
0	0\\
0.0307581521322202	-0.0596435740309786\\
0.0606493011849629	-0.114090598966456\\
0.094332544901734	-0.171426969354389\\
0.128015788618507	-0.224651915623152\\
0.157523732279078	-0.268019711761189\\
0.19408412378904	-0.317687673221434\\
0.230644515299	-0.363034682169355\\
0.267204906808962	-0.404248539976688\\
0.307423722456617	-0.445033241554381\\
0.347642538104273	-0.481282818953179\\
0.387861353751928	-0.51323513242434\\
0.429537038275953	-0.542059845127206\\
0.472669591676345	-0.567565863797025\\
0.515802145076737	-0.588941168677472\\
0.558934698477131	-0.606450982695096\\
0.581814200366589	-0.614258310157062\\
0.627573204145504	-0.62697645369845\\
0.673332207924419	-0.636074134511043\\
0.719091211703335	-0.641831585614293\\
0.767319242644827	-0.644580586248654\\
0.815547273586319	-0.644219481039043\\
0.863775304527811	-0.641035845495326\\
0.913301097684426	-0.635116308519775\\
0.964124653056164	-0.626541825568856\\
1.0149482084279	-0.615719716805604\\
1.06577176379964	-0.602920356406676\\
1.11962318157811	-0.587483336478122\\
1.17347459935659	-0.570393787565765\\
1.25425172602429	-0.54223026779851\\
1.34058328294898	-0.509577897161336\\
1.45569202551523	-0.463391031223006\\
1.83403685224214	-0.308527229685046\\
1.92951697700442	-0.271949408353565\\
2.0249971017667	-0.237223179410069\\
2.12047722652897	-0.204602174551889\\
2.19922805317088	-0.179406965924576\\
2.2943080755078	-0.151142946647814\\
2.37595405398286	-0.128794683011153\\
2.45778879895274	-0.108185369469282\\
2.54327833284418	-0.0885522813547244\\
2.6341792626824	-0.0697563244616575\\
2.71470572836485	-0.0548390107055265\\
2.79667978161753	-0.0412583055701781\\
2.8829965975809	-0.0286240536006428\\
2.96931341354427	-0.0176017551903165\\
3.06581505326621	-0.00705810906366722\\
3.17250151674671	0.00258967028183577\\
3.2791879802272	0.010330689381103\\
3.38643770395599	0.0164006936069949\\
3.50429816540641	0.0213364635919397\\
3.63041644242932	0.0248929840545635\\
3.78634331745148	0.0272593744582608\\
3.95112319705112	0.0278481035671305\\
4.150147525552	0.0266714244115622\\
4.41780544555478	0.0230822709017708\\
5.60154427531905	0.00465262169749181\\
6.00574777816539	0.0013500385829861\\
6.46967388218016	-0.00057467006928924\\
7.09764557760097	-0.00120311836947273\\
9.05067629400935	-8.57398117730668e-05\\
10	4.93971132176085e-05\\
};
\addplot [color=black, dashed, forget plot]
  table[row sep=crcr]{%
0	0\\
0.00245382140710859	-0.00135235075214446\\
0.00526689182363604	-0.00606364780946578\\
0.00883264048603571	-0.0164802726465734\\
0.0129781131986135	-0.034203758807692\\
0.0183267855077567	-0.0648500372319774\\
0.0255618725560876	-0.117929001063194\\
0.0350646588501622	-0.203361308390337\\
0.0472103087424998	-0.330435579313438\\
0.067405706387083	-0.564437415973401\\
0.10521310125109	-1.00342429741558\\
0.124782065914337	-1.20783553432711\\
0.141119346727919	-1.36121639548492\\
0.158053317177576	-1.50232949242482\\
0.17337016863514	-1.614113801331\\
0.186851026214196	-1.70027183558687\\
0.200588975069993	-1.77670967399778\\
0.214536010983682	-1.84311369589632\\
0.226262458583786	-1.89066797350235\\
0.238046239690341	-1.93125934900437\\
0.249830020796898	-1.96504590674533\\
0.259257045682141	-1.98745130266165\\
0.268684070567385	-2.00597199347538\\
0.278993833066806	-2.02202082063134\\
0.287379419659059	-2.0320090175705\\
0.296114113847603	-2.03964787283196\\
0.303147949277808	-2.04384916695491\\
0.31114741428561	-2.04661734711868\\
0.319146879293411	-2.04735109621085\\
0.328362548795928	-2.04581181295651\\
0.337578218298443	-2.0418680435847\\
0.347577192350258	-2.0350429363792\\
0.357967818676723	-2.02533348450613\\
0.368830168807055	-2.01254056037121\\
0.380635966545118	-1.99581618367698\\
0.392441764283182	-1.97640372777905\\
0.406338516717465	-1.95043575841728\\
0.424867519963177	-1.91117091626179\\
0.445691892888979	-1.86161903788522\\
0.467494035834292	-1.80465682683015\\
0.492229488838138	-1.73502316462089\\
0.524458294019663	-1.63839223965513\\
0.571729788586421	-1.48926773233663\\
0.698230137330485	-1.08537053872102\\
0.743489755620585	-0.948684901935895\\
0.780738829896167	-0.841486806522868\\
0.81350711414275	-0.751651652511256\\
0.846578356681519	-0.665514470459748\\
0.879952557512475	-0.583375011601147\\
0.913868605516047	-0.504920870066007\\
0.947965269243825	-0.431168445428057\\
0.982693515445602	-0.361283197354744\\
1.0175491972387	-0.296355205897072\\
1.043977688664	-0.250522761742067\\
1.07040618008929	-0.207540555604499\\
1.09710553694488	-0.166928771027539\\
1.12380489380046	-0.129051536153076\\
1.15050425065605	-0.0938126666834496\\
1.17720360751164	-0.0611122418978383\\
1.20390296436723	-0.030847829859356\\
1.23308587668467	-0.000432142613531639\\
1.26321448016593	0.028190794210639\\
1.29153332669318	0.0526532932029316\\
1.32406813696267	0.0780003198214576\\
1.35660294723217	0.100574516185819\\
1.38913775750166	0.120555088949439\\
1.42160673911414	0.138082828505659\\
1.44949855687578	0.151342790954301\\
1.47930700710908	0.163800094297821\\
1.5135337862116	0.176072060715205\\
1.54923334160385	0.186737703992767\\
1.58957142299652	0.19639853622124\\
1.63100317715634	0.203937674608973\\
1.67571594961762	0.209657633286719\\
1.72302424608013	0.213303120762662\\
1.77292806654387	0.214819725194303\\
1.82679252553232	0.21416658443203\\
1.88197719736234	0.211432327972471\\
1.94337384944909	0.206387716094458\\
2.00642352039961	0.199471510161768\\
2.0914344298269	0.188096667822698\\
2.20062205673976	0.171280171178056\\
2.64311219287125	0.100146198873436\\
2.77217181590961	0.0824255073539462\\
2.88279434994248	0.0689138192197358\\
2.99341688397535	0.0569878648782893\\
3.12247650701371	0.0450235660162388\\
3.25153613005206	0.03502239354202\\
3.38059575309041	0.0267945614338281\\
3.52809246513425	0.0192902168483879\\
3.69402626618356	0.0128853027295239\\
3.87839715623835	0.00779543702188157\\
4.08182358289	0.00405489510939816\\
4.34107671206136	0.00127457334150449\\
4.67231217719191	-0.000303909126634139\\
5.20646596431158	-0.000751501843033964\\
10	3.56587406713516e-07\\
};
\addplot [color=black, dotted, forget plot]
  table[row sep=crcr]{%
0	0\\
0.000571234322185532	-0.0062591722407852\\
0.00174307975931853	-0.0519740816188019\\
0.00416229551319347	-0.235663350450409\\
0.0178607808511906	-1.41375890581984\\
0.0228838527495352	-1.6475250241793\\
0.0276138542085729	-1.78912176957774\\
0.0322439662681262	-1.87659701655303\\
0.0367889155944088	-1.92929523060903\\
0.0410548645421507	-1.9587577759888\\
0.0448590599657628	-1.97396076525753\\
0.0483933489932209	-1.98158427731125\\
0.0517524749113889	-1.98468968416736\\
0.0549793792923126	-1.98491673286712\\
0.0589071044106824	-1.98258065541404\\
0.0640713298760378	-1.97654587550741\\
0.0717585655678654	-1.96386467553035\\
0.0841534480775561	-1.93910658960344\\
0.115057521999143	-1.87182749072639\\
0.197578500177165	-1.69251527434434\\
0.263025187923699	-1.55461873352383\\
0.315934114123497	-1.44645414708083\\
0.365611708758948	-1.34786914496296\\
0.427493576431221	-1.22939311529097\\
0.473904977185425	-1.143858471495\\
0.53693605001204	-1.03245367026655\\
0.578956765229782	-0.961314383719918\\
0.620977480447525	-0.892727629685959\\
0.676849601929167	-0.80553118525048\\
0.729142247738324	-0.728082501085945\\
0.77193338597176	-0.667702534445507\\
0.814724524205195	-0.610009858697481\\
0.860127199175489	-0.551713782378689\\
0.906835642514213	-0.494842664031736\\
0.956230819989365	-0.438072723554242\\
0.992743250488036	-0.39830067638602\\
1.02925568098671	-0.360358940607457\\
1.06576811148538	-0.324215246582058\\
1.10599333991842	-0.286436249169354\\
1.14621856835146	-0.250748005591406\\
1.1864437967845	-0.217098354204557\\
1.22727611937545	-0.184969756783001\\
1.26871553612431	-0.154392883703579\\
1.31015495287317	-0.12579699934728\\
1.35159436962203	-0.099116893806249\\
1.39303378637089	-0.0742861949839462\\
1.43447320311976	-0.0512376891745561\\
1.48510130169825	-0.025398452732917\\
1.52970680110856	-0.00465477861349051\\
1.57130100093479	0.0130601873223473\\
1.61373640207387	0.0295867585771585\\
1.65785420583867	0.0451893061554021\\
1.70197200960346	0.059261144998608\\
1.75556687685267	0.0744105487171929\\
1.80916174410189	0.0875569581893032\\
1.86667609640958	0.099591503549723\\
1.92615019124649	0.1099482241814\\
1.98713395863053	0.118543824875074\\
2.05113707110879	0.125556472407622\\
2.11514018358706	0.130708158477546\\
2.18315254141944	0.134354810343753\\
2.25116489925183	0.13632845198447\\
2.32176361659996	0.13682313503022\\
2.41761947940798	0.135330726157962\\
2.51768808518723	0.131585213963163\\
2.61916093862355	0.125998128840187\\
2.75454093474218	0.116501605845521\\
2.91830993238301	0.103055568309252\\
3.5113880784421	0.0524580749063244\\
3.69340247752602	0.0394331350762229\\
3.87541687660995	0.0282505886429476\\
4.05743127569387	0.0189674702125391\\
4.23944567477779	0.0115153111979396\\
4.44421187374719	0.00513036846328241\\
4.6489780727166	0.000585760432953109\\
4.87673615106517	-0.00269196412403439\\
5.15780433147262	-0.00477207834622106\\
5.51181797563888	-0.00533665490570634\\
6.03917622859277	-0.00410398972216441\\
7.39191711721017	-0.000435878310685922\\
8.49995859081029	0.000207550589145455\\
10	5.81029676016698e-05\\
};
\end{axis}
\end{tikzpicture}%
    \selectlanguage{greek}
    \caption{Οι $x$-τροχιές του κλειστού βρόγχου υπό ανάδραση εξόδου και καταστάσεων, χωρίς κορεσμό στην είσοδο.}
    \label{fig:hgo_sfb_ofb_mu_comparison_no_sat}
\end{figure}

\begin{figure}[t]
    \selectlanguage{english}
    \centering
    % This file was created by matlab2tikz.
%
\tikzstyle{every node}=[font=\footnotesize]
\begin{tikzpicture}

\begin{axis}[%
width=\figurewidth,
height=0.465\figureheight,
at={(0\figurewidth,0.507\figureheight)},
scale only axis,
xmin=0,
xmax=10,
ymin=-0.05,
ymax=1.05,
xticklabels =\empty,
ylabel style={font=\color{white!15!black}},
ylabel={$x_1(t)$},
axis background/.style={fill=white},
legend style={legend cell align=left, align=left, draw=white!15!black},
xlabel style={font=\footnotesize},ylabel style={font=\footnotesize},ticklabel style={font=\scriptsize},x tick label style={font=\scriptsize},y tick label style={font=\scriptsize},legend style={font=\scriptsize}
]
\addplot [color=black]
  table[row sep=crcr]{%
0	1\\
0.0190086209298173	0.999643229379943\\
0.0367363819427684	0.998683186563794\\
0.0606493011849629	0.996468133364459\\
0.0831047969961443	0.993468283471582\\
0.105560292807326	0.989620515567882\\
0.128015788618507	0.9849658623348\\
0.157523732279078	0.977689649632749\\
0.175803928034059	0.97255646725147\\
0.212364319544019	0.960948263689769\\
0.248924711053981	0.947681869595096\\
0.287314314632789	0.932132274886548\\
0.327533130280445	0.914241164422728\\
0.3677519459281	0.894891740224452\\
0.407970761575756	0.874256841599651\\
0.451103314976148	0.850884329961874\\
0.494235868376542	0.826411187798646\\
0.537368421776934	0.80101559744581\\
0.581814200366589	0.774057918857673\\
0.650452706034962	0.731241120674108\\
0.74320522717408	0.671983143368255\\
0.913301097684426	0.562744547496406\\
0.989536430742033	0.51481616510266\\
1.06577176379964	0.468120814535476\\
1.11962318157811	0.436060318005698\\
1.17347459935659	0.404876844939672\\
1.22732601713506	0.374652226821205\\
1.28302891166586	0.344469790827469\\
1.34058328294898	0.314503345138721\\
1.39813765423211	0.285828120069118\\
1.45569202551523	0.258483068760954\\
1.51324639679836	0.232494589419915\\
1.57080076808148	0.207877725305435\\
1.62835513936461	0.184637305267886\\
1.68590951064773	0.162769024178212\\
1.73855672747987	0.143956466495771\\
1.786296789861	0.127868789322561\\
1.83403685224214	0.112690430270064\\
1.88177691462328	0.0984045446733433\\
1.92951697700442	0.0849919432971813\\
1.97725703938556	0.072431389300732\\
2.0249971017667	0.0606998769842555\\
2.07273716414783	0.0497728928299797\\
2.12047722652897	0.0396246593516523\\
2.1795403465104	0.0281070479677332\\
2.23860346649183	0.0176878309495017\\
2.2943080755078	0.00881916572781449\\
2.34329566259284	0.00175134509642838\\
2.39228324967787	-0.00466484966649539\\
2.45778879895274	-0.0122809809631708\\
2.52509814687654	-0.0190305358460741\\
2.57963870477947	-0.0237542557188153\\
2.6341792626824	-0.0278559945338497\\
2.69457411194424	-0.031722022292314\\
2.77510057762669	-0.0358627727900309\\
2.83983818959922	-0.0384281465803618\\
2.92615500556259	-0.0409046828548885\\
3.01247182152596	-0.0424294314524083\\
3.10137720775971	-0.0431421382886104\\
3.20806367124021	-0.0430313040285224\\
3.3149378881368	-0.0420589733047088\\
3.44430686090904	-0.0399929876025062\\
3.60888542392316	-0.0363782430951662\\
3.83201022720331	-0.0305022725955926\\
4.3352022341174	-0.0170029359532524\\
4.56216383942848	-0.0118842446473248\\
4.79313726962638	-0.00759103173139763\\
5.02411069982429	-0.00424316145569748\\
5.25508413002219	-0.00177445065563298\\
5.51492923899484	9.76014693137017e-05\\
5.80364602674222	0.00128455345961243\\
6.15010617203908	0.00183133877646036\\
6.64981065373861	0.00167220547342595\\
8.86499569950953	-4.4686484683254e-05\\
10	-6.27923251883544e-05\\
};
\addlegendentry{SFB}

\addplot [color=black, dashed]
  table[row sep=crcr]{%
0	1\\
0.0183990670283389	0.999612124968426\\
0.0321803398311165	0.998481392168163\\
0.0474119799138659	0.996347830410389\\
0.0621147833158044	0.993408104620551\\
0.077625694195925	0.989369518763775\\
0.0921499041417331	0.98471536806421\\
0.10720316463996	0.979001213204821\\
0.122911251157451	0.972072085324402\\
0.137178594675865	0.964923156351878\\
0.151911763151949	0.956686277901067\\
0.167112439348932	0.94727800036207\\
0.182781166613184	0.936612659663165\\
0.198917860881512	0.92460231591023\\
0.215518187378088	0.911160005908219\\
0.232569711149353	0.896204702948227\\
0.250509697862251	0.879214665342838\\
0.267937888270239	0.861504280485551\\
0.286540183399881	0.841459569332271\\
0.307907899977979	0.817240262973593\\
0.331987400388602	0.788746928343885\\
0.364158185642644	0.749270948733193\\
0.414840201226497	0.685340329782647\\
0.48577208595816	0.595742957873838\\
0.526499317986902	0.545635198059642\\
0.561626041387312	0.503723791599324\\
0.590228358615279	0.470670093719015\\
0.620096805107693	0.437296415224681\\
0.650280505514027	0.404840380558346\\
0.681409967662125	0.372768694372663\\
0.713172344887708	0.341555890750323\\
0.737468814115008	0.318725949577415\\
0.762078628146281	0.296530865158852\\
0.787315131785499	0.274740261335323\\
0.812551635424718	0.253924803393037\\
0.838722895255389	0.233356346603351\\
0.864894155086057	0.213807881443888\\
0.891687275692821	0.194830393550273\\
0.918791326687629	0.17667336101729\\
0.946206509404895	0.159343738408142\\
0.974243955567072	0.142665408051824\\
1.00228140172925	0.12700805408773\\
1.03125413301423	0.111861756255168\\
1.06022686429921	0.0977244430068751\\
1.08982413110991	0.084280191241902\\
1.11973366568346	0.0716726417826763\\
1.14995519899849	0.0598853009746776\\
1.18080072979647	0.0487917285915955\\
1.21164626059445	0.0385952813055432\\
1.24342228651646	0.0289777079812161\\
1.27519831243848	0.0202081426426126\\
1.30697433836049	0.0122359693251628\\
1.3387503642825	0.00501223439309761\\
1.37052639020452	-0.00151025591837595\\
1.4128944247672	-0.00919422548997417\\
1.45551514869465	-0.0158508070777224\\
1.49838856198685	-0.0215597821391906\\
1.54428138427366	-0.0266795748580329\\
1.59118067622533	-0.0309613829720039\\
1.63890223376613	-0.0344404010069841\\
1.6868182548251	-0.0371451607375892\\
1.74744251934195	-0.0395821490452697\\
1.80836483845797	-0.0410842827234692\\
1.86936167122379	-0.0417934251074037\\
1.94255787054277	-0.0417875875882778\\
2.02795343641492	-0.0408567936147755\\
2.12554836884024	-0.0389054685124801\\
2.25010885941539	-0.0355299416538717\\
2.46740535450567	-0.0286112530846907\\
2.74090043653433	-0.0200985614159226\\
2.9232304912201	-0.0152409186344205\\
3.10556054590588	-0.0112131750749231\\
3.28789060059166	-0.00800537699149473\\
3.48845366074601	-0.00532578977819576\\
3.70724972636894	-0.00324529386598904\\
3.9654788797126	-0.00165349615730825\\
4.28453202674741	-0.000568628508347047\\
4.71775728499171	1.93787479449981e-05\\
5.54401382990565	0.000129649963961143\\
10	-8.84916424581661e-08\\
};
\addlegendentry{$\text{OFB w/ sat, }\mu\text{ = 0.1}$}

\addplot [color=black, dotted]
  table[row sep=crcr]{%
0	1\\
0.0145382169968897	0.999580167149892\\
0.0289949819379949	0.998324345765548\\
0.0437993213022487	0.996171948863472\\
0.0587988521398071	0.993109810308342\\
0.0788930000356967	0.988039751455466\\
0.102459318118321	0.981222144880959\\
0.127802930576522	0.973012208890502\\
0.154539746256198	0.963441775921122\\
0.180886306820776	0.953160479208499\\
0.209858143774763	0.940950804267095\\
0.238117435389102	0.928199249754002\\
0.268131518348953	0.913821133719951\\
0.294183816986855	0.90070276548273\\
0.323276424143726	0.8854154380384\\
0.36691533487903	0.861359766976461\\
0.413316246806419	0.834508010290525\\
0.461098159329852	0.805727006027459\\
0.515166530697094	0.772071161640488\\
0.581807819751949	0.729446487945298\\
0.67248225870628	0.670311449294283\\
0.816681158103126	0.576111803873102\\
0.892166120963692	0.527780434178531\\
0.967651083824258	0.480677065057948\\
1.01954519285428	0.449183889360407\\
1.07301010234062	0.417607365179899\\
1.12647501182695	0.386996980686609\\
1.17993992131329	0.357422727039715\\
1.23502495584754	0.328096013274379\\
1.29010999038178	0.299978384323474\\
1.34519502491603	0.273106311923117\\
1.40028005945028	0.247504981446053\\
1.45536509398453	0.223189355247342\\
1.51045012851877	0.200165182231803\\
1.56553516305302	0.17842995350725\\
1.62062019758727	0.157973804230878\\
1.67570523212152	0.13878036210812\\
1.72608581198201	0.12231296791165\\
1.77176193716875	0.108263101005827\\
1.81743806235548	0.0950319417470737\\
1.86311418754222	0.0826002154574823\\
1.90879031272896	0.0709470971069397\\
1.95446643791569	0.0600504363161232\\
2.00014256310243	0.0498869675051985\\
2.04581868828917	0.0404325057202026\\
2.0914948134759	0.0316621286493355\\
2.14807301584015	0.0217087134237559\\
2.2046512182044	0.0127164838856775\\
2.26122942056865	0.00463559143000403\\
2.31780762293289	-0.00258400026575778\\
2.37438582529714	-0.00899218274464708\\
2.43096402766139	-0.0146384078259949\\
2.48754223002564	-0.0195714278844683\\
2.54429965985844	-0.0238515807100477\\
2.62009571794789	-0.0286048290153982\\
2.7010334804612	-0.032577264624889\\
2.78368514444912	-0.0355862203620241\\
2.86633680843705	-0.0376628134527106\\
2.94898847242498	-0.0389263144170666\\
3.03491629741763	-0.0394970841878308\\
3.13846952389247	-0.0393396531158796\\
3.26081815680652	-0.0381973778155107\\
3.39066034185652	-0.0361314424481449\\
3.55271094056516	-0.0327186241734996\\
3.82028727025877	-0.026083220826127\\
4.21911443389052	-0.0162490514267351\\
4.46864295478243	-0.0110360239812302\\
4.69369637856469	-0.0071912290844125\\
4.91874980234696	-0.00417828283812227\\
5.171934904102	-0.0017075132834794\\
5.42512000585705	-6.791908701409e-05\\
5.73456846355766	0.00105682275091468\\
6.10083899156157	0.00152744186345544\\
6.66652136426931	0.00130558968468897\\
8.5652205547228	3.02333764956586e-06\\
10	-4.40216865378318e-05\\
};
\addlegendentry{$\text{OFB w/ sat, }\mu\text{ = 0.01}$}

\end{axis}

\begin{axis}[%
width=\figurewidth,
height=0.465\figureheight,
at={(0\figurewidth,0\figureheight)},
scale only axis,
xmin=0,
xmax=10,
xlabel style={font=\color{white!15!black}},
xlabel={Time, $t$ [\si{\second}]},
ymin=-1.3,
ymax=0.07,
ylabel style={font=\color{white!15!black}},
ylabel={$x_2(t)$},
axis background/.style={fill=white},
xlabel style={font=\footnotesize},ylabel style={font=\footnotesize},ticklabel style={font=\scriptsize},x tick label style={font=\scriptsize},y tick label style={font=\scriptsize},legend style={font=\scriptsize}
]
\addplot [color=black, forget plot]
  table[row sep=crcr]{%
0	0\\
0.0218942716257455	-0.0428368214562891\\
0.0427146117533166	-0.081832094293933\\
0.0606493011849629	-0.114090598966456\\
0.0831047969961443	-0.15277915936759\\
0.105560292807326	-0.189617961695983\\
0.128015788618507	-0.224651915623152\\
0.157523732279078	-0.268019711761189\\
0.175803928034059	-0.293405641626192\\
0.19408412378904	-0.317687673221434\\
0.212364319544019	-0.340889502138996\\
0.230644515299	-0.363034682169355\\
0.248924711053981	-0.384146614934043\\
0.267204906808962	-0.404248539976688\\
0.287314314632789	-0.425222833950917\\
0.307423722456617	-0.445033241554381\\
0.327533130280445	-0.463709940858665\\
0.347642538104273	-0.481282818953179\\
0.3677519459281	-0.49778145973843\\
0.387861353751928	-0.51323513242434\\
0.407970761575756	-0.527672780725444\\
0.429537038275953	-0.542059845127206\\
0.451103314976148	-0.555346080540048\\
0.472669591676345	-0.567565863797025\\
0.494235868376542	-0.578753091343955\\
0.515802145076737	-0.588941168677472\\
0.537368421776934	-0.598163000660941\\
0.558934698477131	-0.606450982695096\\
0.581814200366589	-0.614258310157062\\
0.604693702256046	-0.621087885728741\\
0.627573204145504	-0.62697645369845\\
0.650452706034962	-0.63196007774811\\
0.673332207924419	-0.636074134511043\\
0.696211709813877	-0.639353308106367\\
0.719091211703335	-0.641831585614293\\
0.74320522717408	-0.643613367432872\\
0.767319242644827	-0.644580586248654\\
0.791433258115573	-0.644770496511939\\
0.815547273586319	-0.644219481039043\\
0.839661289057064	-0.642963050766845\\
0.863775304527811	-0.641035845495326\\
0.887889319998557	-0.638471635570603\\
0.913301097684426	-0.635116308519775\\
0.938712875370294	-0.631127568458417\\
0.964124653056164	-0.626541825568856\\
1.0149482084279	-0.615719716805604\\
1.06577176379964	-0.602920356406676\\
1.11962318157811	-0.587483336478122\\
1.17347459935659	-0.570393787565765\\
1.22732601713506	-0.551914364115015\\
1.28302891166586	-0.531596161353969\\
1.34058328294898	-0.509577897161336\\
1.42691483987367	-0.475135853258646\\
1.54202358243992	-0.427718336454053\\
1.7146866962893	-0.356319435928011\\
1.81016682105157	-0.317922827162072\\
1.88177691462328	-0.2900261179262\\
1.95338700819499	-0.26308290104056\\
2.0249971017667	-0.237223179410069\\
2.0966071953384	-0.212549264368691\\
2.15985263984993	-0.19180594054977\\
2.21891575983136	-0.173358669178558\\
2.27797887981279	-0.155826583015454\\
2.34329566259284	-0.137519680381699\\
2.40865963699659	-0.120343463521838\\
2.47416518627146	-0.104275325976575\\
2.54327833284418	-0.0885522813547244\\
2.61599907671476	-0.0733471246813586\\
2.67444249552362	-0.0620981075562757\\
2.73483734478546	-0.0513565705019374\\
2.79667978161753	-0.0412583055701781\\
2.86141739359006	-0.0316275182031731\\
2.92615500556259	-0.0229178477016667\\
2.99089261753512	-0.0150858576442285\\
3.06581505326621	-0.00705810906366722\\
3.13693936225321	-0.000404045150842691\\
3.20806367124021	0.00537153420914827\\
3.297062934182	0.011455927904759\\
3.38643770395599	0.0164006936069949\\
3.48430106390729	0.0206161373738531\\
3.58735440541699	0.0238608265711644\\
3.69500949794782	0.0261192535209034\\
3.80917677232739	0.0274460433239998\\
3.92705331830814	0.0278640712572908\\
4.07405871018882	0.027318319612748\\
4.25460493512343	0.0254734340839455\\
4.504420481879	0.0216433626875574\\
5.34169916634641	0.0077411221083814\\
5.63041595409379	0.00435750696001236\\
5.94800442061591	0.00171775436186294\\
6.29447387076817	-4.16877346012257e-05\\
6.71157974765967	-0.00101089468880389\\
7.26839972642691	-0.00116212475522559\\
10	4.93971132176085e-05\\
};
\addplot [color=black, dashed, forget plot]
  table[row sep=crcr]{%
0	0\\
0.00167446494274337	-0.000634497027864001\\
0.00357904957371957	-0.00284590367247439\\
0.00582950590694153	-0.00738822925755933\\
0.00883264048603571	-0.0164802726465734\\
0.0161862438835207	-0.0456346646259345\\
0.260940211728625	-1.02296023874388\\
0.271436726541047	-1.0594182234637\\
0.283175593169084	-1.09589980935052\\
0.293269363861477	-1.12385912300004\\
0.303953222150525	-1.1502235964417\\
0.315817255632886	-1.17584392104964\\
0.323726611287796	-1.19090846435691\\
0.331987400388602	-1.20501218251648\\
0.340599622935306	-1.21802649271848\\
0.349211845482012	-1.22940077789487\\
0.357824068028718	-1.23921719639271\\
0.364158185642644	-1.24548865314868\\
0.370492303256572	-1.25099018380751\\
0.376826420870501	-1.25575098260331\\
0.383160538484429	-1.25979932324614\\
0.389494656098357	-1.26316258618691\\
0.395828773712285	-1.2658672851815\\
0.402162891326213	-1.26793909313545\\
0.408501546276355	-1.26940370362898\\
0.414840201226497	-1.2702835260296\\
0.421178856176638	-1.27060190781217\\
0.433856166076923	-1.26964402838334\\
0.446533475977207	-1.26670231094231\\
0.459613012637524	-1.26175880859432\\
0.472692549297843	-1.2550410261192\\
0.48577208595816	-1.24670381540124\\
0.499149347028036	-1.23665235531114\\
0.512824332507469	-1.22493278473934\\
0.526499317986902	-1.2118953715798\\
0.540174303466335	-1.19767361526297\\
0.554475462080319	-1.18166837833479\\
0.575927200001296	-1.1557714696214\\
0.597695470238383	-1.12754011091602\\
0.620096805107693	-1.09682705725796\\
0.650280505514027	-1.05340511470711\\
0.68919233319915	-0.995084461319628\\
0.75366646026654	-0.895864435154699\\
0.821275388701608	-0.792453746095116\\
0.864894155086057	-0.727704964102793\\
0.900721959357757	-0.676224269494123\\
0.936860694017502	-0.626127342148111\\
0.974243955567072	-0.576435404259515\\
1.00228140172925	-0.540673301582901\\
1.03125413301423	-0.50512263823583\\
1.06022686429921	-0.471025919421198\\
1.08982413110991	-0.437708947758646\\
1.11973366568346	-0.40559680769401\\
1.14995519899849	-0.374728761485374\\
1.18080072979647	-0.344839719681014\\
1.21164626059445	-0.316554364767068\\
1.24342228651646	-0.289054420440298\\
1.27519831243848	-0.263172243964631\\
1.30697433836049	-0.238857404960305\\
1.3387503642825	-0.216055887575534\\
1.37052639020452	-0.194711093874156\\
1.40230241612653	-0.174764690617375\\
1.43407844204854	-0.156157321467463\\
1.4662335020177	-0.138629987191885\\
1.49838856198685	-0.122350733658193\\
1.53255656128574	-0.106351473755618\\
1.56773103024949	-0.0912041775148698\\
1.60311106561053	-0.077245060146625\\
1.63890223376613	-0.0643483233297655\\
1.67469340192173	-0.0526065694827498\\
1.71106796063184	-0.0417803008059021\\
1.74744251934195	-0.0319953428645974\\
1.78396610535164	-0.0231457362788792\\
1.82056420501114	-0.0151890148660776\\
1.85716230467063	-0.00807781412718356\\
1.90595977088328	0.000197001459191881\\
1.95475723709594	0.00722058441032658\\
2.00355470330859	0.0131228980396294\\
2.05235216952125	0.0180239607755048\\
2.10114963573391	0.0220342415430732\\
2.16214646849973	0.0259480178753204\\
2.2249767477252	0.0288781333848807\\
2.29635105753488	0.0310684021119449\\
2.37624032716278	0.0323467382699416\\
2.46740535450567	0.0326196474706553\\
2.57680338731713	0.031720353182676\\
2.70443442559717	0.0295446480505763\\
2.90499748575153	0.0248643816476264\\
3.34258961699739	0.014280580306762\\
3.5796186880889	0.00977068004563364\\
3.81680936200972	0.00635215558711799\\
4.06145787793918	0.00385033717625127\\
4.34806560789733	0.00196043442657867\\
4.71775728499171	0.000656471640814615\\
5.22193340137524	2.94224514085784e-06\\
6.32663654982034	-7.96121663864113e-05\\
10	1.77433371106872e-07\\
};
\addplot [color=black, dotted, forget plot]
  table[row sep=crcr]{%
0	0\\
0.00160827527504104	-0.00623408945712178\\
0.0560048637769412	-0.221777031558357\\
0.0612435919573144	-0.236400564087749\\
0.0672202619263267	-0.249882573226676\\
0.0745622312705176	-0.263764835787747\\
0.0842582245938637	-0.27973398041954\\
0.0977693146348706	-0.299882104911811\\
0.11686624774779	-0.326394741083758\\
0.138778015811482	-0.355024833924784\\
0.161571562209947	-0.38306361290965\\
0.184803113553384	-0.40989905037099\\
0.206754475498032	-0.433690972454183\\
0.229623468568651	-0.456909352081418\\
0.250858385619777	-0.477078441958112\\
0.274644593008428	-0.498130416368369\\
0.294183816986855	-0.514242524456913\\
0.323276424143726	-0.536326065390965\\
0.352369031300595	-0.556214201322756\\
0.381461638457465	-0.574001102017556\\
0.413316246806419	-0.59117581466049\\
0.445170855155375	-0.60606031426382\\
0.477025463504331	-0.618770095535991\\
0.492952767678808	-0.624344756613752\\
0.515166530697094	-0.631284901719525\\
0.53738029371538	-0.637285920585237\\
0.559594056733664	-0.642383904093496\\
0.581807819751949	-0.646614225355512\\
0.604021582770235	-0.650011536696024\\
0.626235345788521	-0.652609767431871\\
0.648449108806805	-0.654442122455778\\
0.67248225870628	-0.655599578073193\\
0.696515408605753	-0.655938747042208\\
0.720548558505229	-0.65549881405083\\
0.744581708404702	-0.654317974387649\\
0.768614858304177	-0.652433437407661\\
0.792648008203651	-0.649881430960933\\
0.816681158103126	-0.646697206727255\\
0.841842812389981	-0.642723315781252\\
0.867004466676837	-0.638132218915407\\
0.892166120963692	-0.632960979654685\\
0.942489429537403	-0.621020658844227\\
0.992812738111112	-0.607176355053012\\
1.04627764759745	-0.59066827723616\\
1.09974255708378	-0.572586688725748\\
1.15320746657012	-0.553196845482201\\
1.20748243858041	-0.532426723869419\\
1.29010999038178	-0.499227283654362\\
1.37273754218316	-0.46477140501994\\
1.67570523212152	-0.337097213838829\\
1.77176193716875	-0.29856826596424\\
1.84027612494885	-0.272152188851546\\
1.90879031272896	-0.246761257015324\\
1.97730450050906	-0.222489542130628\\
2.04581868828917	-0.199408614587817\\
2.11035421426399	-0.17880341087041\\
2.16693241662823	-0.161668262499612\\
2.22351061899248	-0.145413276911571\\
2.28008882135673	-0.130042788887252\\
2.33666702372098	-0.115555143929557\\
2.39324522608523	-0.101943368353984\\
2.44982342844947	-0.0891957987054344\\
2.50640163081372	-0.0772966710609886\\
2.56324867438081	-0.0661760145959462\\
2.62009571794789	-0.0558697016319734\\
2.68037056446421	-0.0458031928429996\\
2.74235931245516	-0.0363411485266525\\
2.80434806044611	-0.0277445585991511\\
2.86633680843705	-0.0199734850358535\\
2.94898847242498	-0.0108246989499126\\
3.0177307324191	-0.00420554733196887\\
3.10387847565358	0.00291490176127418\\
3.1903560962508	0.00885569049497548\\
3.27843367194545	0.0137808591777517\\
3.37141143194821	0.017875902053337\\
3.46926784537817	0.0211033197635775\\
3.57357171436191	0.0234821114320169\\
3.68326205801066	0.0249790702231873\\
3.82028727025877	0.0256640166586166\\
3.96371212991288	0.0252800460936253\\
4.14020427126687	0.0237000600537005\\
4.38424792086408	0.020297509160665\\
5.25632993802035	0.00694880672279119\\
5.56577839572097	0.00371580304411978\\
5.90335853139436	0.00131530274478742\\
6.27395046103295	-0.000194683371878668\\
6.72900426922911	-0.000928248373952556\\
7.42924613721526	-0.000851782662255829\\
9.82305229623582	3.49351028408051e-05\\
10	3.86793456019774e-05\\
};
\end{axis}
\end{tikzpicture}%
    \selectlanguage{greek}
    \caption{Οι $x$-τροχιές του κλειστού βρόγχου υπό ανάδραση εξόδου και καταστάσεων, με κορεσμό στην είσοδο.}
    \label{fig:hgo_sfb_ofb_mu_comparison_sat}
\end{figure}

\begin{figure}[t]
    \selectlanguage{english}
    \centering
    % This file was created by matlab2tikz.
%
\tikzstyle{every node}=[font=\footnotesize]
\begin{tikzpicture}

\begin{axis}[%
width=\figurewidth,
height=0.465\figureheight,
at={(0\figurewidth,0.507\figureheight)},
scale only axis,
xmin=0,
xmax=0.15,
ymin=-110,
ymax=10,
xticklabels = \empty,
ylabel style={font=\color{white!15!black}},
ylabel={$u(\hat x(t))$},
axis background/.style={fill=white},
xlabel style={font=\footnotesize},ylabel style={font=\footnotesize},ticklabel style={font=\scriptsize},x tick label style={font=\scriptsize},y tick label style={font=\scriptsize},legend style={font=\scriptsize}
]
\addplot [color=black, forget plot]
  table[row sep=crcr]{%
0	0\\
0.000522260809063368	-19.6135026152555\\
0.00105375913702233	-36.5644318909725\\
0.00155906864006283	-50.1908645880645\\
0.00204245836555117	-61.2149451246212\\
0.00256870505856455	-71.2377382058122\\
0.00302244123211892	-78.4103465527967\\
0.00351588144251025	-84.8438941199955\\
0.00402255633183302	-90.1384990695971\\
0.00444177387591083	-93.6293954290879\\
0.00486099141998864	-96.4034375109122\\
0.00530646308395433	-98.6497180258689\\
0.00561219556657022	-99.8149077744467\\
0.00591792804917191	-100.701919855461\\
0.00623705496258253	-101.355598200319\\
0.00640331563468521	-101.593924069681\\
0.00656957630680211	-101.766447260281\\
0.00673583697890479	-101.87625775961\\
0.00690209765100747	-101.926329000782\\
0.00706835832311015	-101.919521941629\\
0.00723461899521283	-101.858589006605\\
0.00741206992582022	-101.736826625336\\
0.0075895208564134	-101.559514261122\\
0.00794442271761397	-101.049982293827\\
0.00829932457881455	-100.352290947734\\
0.00866120942578164	-99.4683578410431\\
0.00921451117490335	-97.824691236502\\
0.00976781292402507	-95.8835907043156\\
0.0105333267794521	-92.8100866502061\\
0.0115040274684901	-88.4407591815262\\
0.0127006490627224	-82.6059686550782\\
0.0174071400561928	-59.1684774847924\\
0.0187757345107684	-52.9638401875392\\
0.0199481968461583	-47.9942185961442\\
0.0211526905459607	-43.240610536779\\
0.0223820221100226	-38.7612003397801\\
0.0233856833890513	-35.3791779058015\\
0.0244149761437598	-32.1601119715555\\
0.0254528127236995	-29.1600069294009\\
0.0265291860529544	-26.296476737383\\
0.0276138542085675	-23.6518620270794\\
0.0287234068432838	-21.1808884751911\\
0.0298675312883745	-18.8636300475277\\
0.0307515534799876	-17.2223141379358\\
0.0316419231612031	-15.6916257379833\\
0.0325449878215949	-14.2564230660171\\
0.0334480524819867	-12.931392267122\\
0.0343959908682905	-11.6506859519549\\
0.0353439292545943	-10.474569069564\\
0.0364125799684984	-9.26423425859342\\
0.0373693726181159	-8.27616119987104\\
0.0382400581536899	-7.44924796840884\\
0.0391398771061233	-6.66150895135975\\
0.0400542627670006	-5.92537127549485\\
0.0410548645421471	-5.18824187306629\\
0.0422278985458888	-4.40714879715428\\
0.0434009325496305	-3.70715602890614\\
0.0444602906592308	-3.13786685345171\\
0.0452578292723018	-2.74517022870647\\
0.0463607800067081	-2.24849955300418\\
0.047445519163972	-1.80797027428012\\
0.0483933489932156	-1.45836711841255\\
0.0492284095034847	-1.17533257166687\\
0.0501255640829186	-0.895236163009869\\
0.0510666030610452	-0.626003030731795\\
0.0520954108365572	-0.357974396183593\\
0.0531242186120835	-0.114971352936678\\
0.0542373150202167	0.122423729733001\\
0.0553504114283641	0.335876937957337\\
0.0565120457344221	0.53569147439822\\
0.0573026479044501	0.659519894443349\\
0.058093250074478	0.774316246008411\\
0.0589071044106788	0.883742727123973\\
0.059744210913081	0.987727269999638\\
0.0605813174154832	1.08368151491423\\
0.0614184239178854	1.17222023009155\\
0.0623027259039333	1.25832419726082\\
0.0631870278899811	1.33740381107332\\
0.0640713298760431	1.4100269444066\\
0.0649804047172893	1.47850337404552\\
0.0659142524137195	1.54286278268131\\
0.066848100110164	1.60167143390905\\
0.0677819478066084	1.65540296240549\\
0.068768887562058	1.70715035032519\\
0.0697558273175076	1.75417746200436\\
0.0707427670729572	1.79690929761631\\
0.0717585655678619	1.83681275265631\\
0.0728032228022357	1.87393455151418\\
0.0738478800366096	1.9074603802561\\
0.0748925372709692	1.93773226634954\\
0.0760006519684424	1.9666312187308\\
0.0771087666659156	1.99255017986013\\
0.0782168813633746	2.01578897296496\\
0.079360223445974	2.03724262337667\\
0.0805387929136998	2.05695210537695\\
0.0817173623814256	2.07447779336334\\
0.0828959318491513	2.09005255340357\\
0.0841534480775579	2.10475332656515\\
0.0854109643059644	2.11769444724194\\
0.0866684805343851	2.12907526652081\\
0.0872972386485884	2.13423659815587\\
0.0879706365505513	2.1394039636605\\
0.0886440344525283	2.14422129135248\\
0.0893174323544912	2.14871038707922\\
0.089990830256454	2.15289169316642\\
0.0906642281584311	2.15678437358989\\
0.091337626060394	2.16040639394409\\
0.092011023962371	2.16377459638666\\
0.0927353904755535	2.16713242159528\\
0.0934597569887217	2.17023318660239\\
0.0941841235019041	2.17309403776544\\
0.0949084900150865	2.17573097059086\\
0.0956328565282689	2.17815890692469\\
0.0963572230414513	2.18039176708457\\
0.0970815895546338	2.18244253711954\\
0.0978647783119015	2.18446894907981\\
0.0986479670691693	2.18631089051871\\
0.0994311558264371	2.18798157599463\\
0.100214344583719	2.18949326364822\\
0.100997533340987	2.19085732439369\\
0.101780722098255	2.19208430622696\\
0.102563910855523	2.19318399384336\\
0.103415749760586	2.19424614278678\\
0.10426758866565	2.19517922855648\\
0.105119427570713	2.19599320344656\\
0.105971266475777	2.19669723888506\\
0.10682310538084	2.1972997866889\\
0.107674944285904	2.19780863564114\\
0.108526783190968	2.19823096358851\\
0.109459745877842	2.19860205408825\\
0.110392708564731	2.19888526264718\\
0.11132567125162	2.19908788561639\\
0.112258633938495	2.1992165949198\\
0.113191596625384	2.19927749149961\\
0.114124559312259	2.1992761543207\\
0.115057521999148	2.19921768513623\\
0.116087767040597	2.19909233658487\\
0.117118012082059	2.1989087197109\\
0.118148257123508	2.19867201652066\\
0.120208747206419	2.19805768929997\\
0.122269237289345	2.19728186393233\\
0.124567062021328	2.19625871418093\\
0.12686488675331	2.19509787222772\\
0.129162711485293	2.19382348518968\\
0.131608242991518	2.19236413122461\\
0.134201481271973	2.1907207880294\\
0.136794719552427	2.18899589791872\\
0.140872964922167	2.18615019216404\\
0.145327986189983	2.18289453246226\\
0.149783007457799	2.17951831370047\\
0.151515596140939	2.17817785887678\\
};
\end{axis}

\begin{axis}[%
width=\figurewidth,
height=0.465\figureheight,
at={(0\figurewidth,0.0\figureheight)},
scale only axis,
xmin=0,
xmax=0.15,
xlabel style={font=\color{white!15!black}},
xlabel={Time, $t$ [\si{\second}]},
ymin=-4.2,
ymax=0.1,
ylabel style={font=\color{white!15!black}},
ylabel={$u_s(\hat x(t))$},
axis background/.style={fill=white},
xlabel style={font=\footnotesize},ylabel style={font=\footnotesize},ticklabel style={font=\scriptsize},x tick label style={font=\scriptsize},y tick label style={font=\scriptsize},legend style={font=\scriptsize}
]
\addplot [color=black, forget plot]
  table[row sep=crcr]{%
0	0\\
0.000138652339244594	-4\\
0.0515593222251196	-4\\
0.0515708549418772	-3.99818800249206\\
0.0525301519687371	-3.78078967101512\\
0.0533649719890423	-3.60708272781904\\
0.054204178323932	-3.44579327684629\\
0.0552331414399374	-3.26474511871557\\
0.0560048637769421	-3.1400489973091\\
0.0570526094130166	-2.98462694144009\\
0.0581003550490911	-2.84377292676686\\
0.0591481006851655	-2.71608115829852\\
0.06019584632124	-2.60028272770953\\
0.0612435919573144	-2.49523206066568\\
0.0622913375933889	-2.39989474741859\\
0.0633390832294642	-2.31333660840023\\
0.0643868288655387	-2.23471386301191\\
0.0654345745016132	-2.16326428516965\\
0.0663583985987133	-2.10566574723391\\
0.0672202619263267	-2.05609629358793\\
0.068304410833961	-1.99895148687547\\
0.0694158387340611	-1.9458324231065\\
0.0704059329960849	-1.90271845475002\\
0.0714296779054227	-1.86190943555684\\
0.0723375228860048	-1.82864072680008\\
0.0732453678665879	-1.79787830581226\\
0.0742241484670494	-1.76727983686472\\
0.0752383968774515	-1.73813437219801\\
0.0762832730541918	-1.71058312574186\\
0.0773894047636068	-1.68388980064849\\
0.0784955364730227	-1.65948866208011\\
0.0796879271610456	-1.63548751792829\\
0.0808803178490685	-1.6136237690275\\
0.0821293366066342	-1.59276283474143\\
0.0834066693989719	-1.57334754442527\\
0.084712879928075	-1.55526893865663\\
0.0860768459307115	-1.5380808959934\\
0.0874408119333481	-1.52241127190443\\
0.0884105833192326	-1.51209101412547\\
0.0898652403980584	-1.49774452799521\\
0.0913522063102175	-1.4843220445677\\
0.0923865953627665	-1.47563767681913\\
0.0939381789415901	-1.4634938364258\\
0.0950076969956983	-1.45567233875335\\
0.0966646675792022	-1.44432811603143\\
0.0983216381627052	-1.43380298693975\\
0.100094929572254	-1.42331214834543\\
0.101868220981804	-1.4135002909168\\
0.10372826433689	-1.40382454468828\\
0.105631683664743	-1.39447283232107\\
0.107584256911884	-1.38536358484279\\
0.109635137997598	-1.37623575084047\\
0.111686019083313	-1.36748263549022\\
0.113906117082375	-1.35835617761654\\
0.116126215081438	-1.34952847175393\\
0.119283071569993	-1.33738796928968\\
0.122505503332929	-1.3253856622632\\
0.126037121495323	-1.31257586822334\\
0.129661253706632	-1.29971492715382\\
0.134538346654193	-1.28274645807883\\
0.139866099334075	-1.26453841827217\\
0.145586541322665	-1.24527206204819\\
0.150498924164292	-1.22891235224872\\
};
\end{axis}
\end{tikzpicture}%
    \selectlanguage{greek}
    \caption{Σύγκριση των νόμων ελέγχου ανάδρασης εξόδου με~\eqref{eq:hgo ofb control w/ sat} και χωρίς~\eqref{eq:hgo ofb control w/o sat} κορεσμό, με $\mu = 0.01$.}
    \label{fig:hgo_u_comparison}
\end{figure}

\begin{figure}[t]
    \selectlanguage{english}
    \centering
    % This file was created by matlab2tikz.
%
\tikzstyle{every node}=[font=\footnotesize]
\begin{tikzpicture}

\begin{axis}[%
width=\figurewidth,
height=0.465\figureheight,
at={(0\figurewidth,0.507\figureheight)},
scale only axis,
xmin=0,
xmax=10,
ymin=-0.1,
ymax=1.05,
xticklabels = \empty,
ylabel style={font=\color{white!15!black}},
ylabel={$x_1(t)$},
axis background/.style={fill=white},
legend style={legend cell align=left, align=left, draw=white!15!black},
xlabel style={font=\footnotesize},ylabel style={font=\footnotesize},ticklabel style={font=\scriptsize},x tick label style={font=\scriptsize},y tick label style={font=\scriptsize},legend style={font=\scriptsize}
]
\addplot [color=black]
  table[row sep=crcr]{%
0	1\\
0.0190086209298173	0.999643229379943\\
0.0367363819427684	0.998683186563794\\
0.0606493011849629	0.996468133364459\\
0.0831047969961443	0.993468283471582\\
0.105560292807326	0.989620515567882\\
0.128015788618507	0.9849658623348\\
0.157523732279078	0.977689649632749\\
0.19408412378904	0.966969351042962\\
0.230644515299	0.954512736001034\\
0.267204906808962	0.940474339768624\\
0.307423722456617	0.923380182721619\\
0.347642538104273	0.904737717184577\\
0.387861353751928	0.884724540991085\\
0.429537038275953	0.862719757185671\\
0.472669591676345	0.838773929160828\\
0.515802145076737	0.813818013151536\\
0.558934698477131	0.788024428749029\\
0.627573204145504	0.74564473244285\\
0.696211709813877	0.702141369054495\\
0.964124653056164	0.53067345720309\\
1.04035998611377	0.483527284573485\\
1.09269747268888	0.451986613503758\\
1.14654889046735	0.420353507593624\\
1.20040030824582	0.389640108853536\\
1.25425172602429	0.359921307038547\\
1.31180609730742	0.329328124423034\\
1.36936046859054	0.300001586732961\\
1.42691483987367	0.271987392901766\\
1.4844692111568	0.24531806491693\\
1.54202358243992	0.220014177212095\\
1.59957795372305	0.19608552701839\\
1.65713232500617	0.173532244916727\\
1.7146866962893	0.15234584508018\\
1.76242675867043	0.135798010425626\\
1.81016682105157	0.120166939268685\\
1.85790688343271	0.10543707623491\\
1.90564694581385	0.0915903661244819\\
1.95338700819499	0.0786065599188941\\
2.00112707057612	0.0664635026285918\\
2.04886713295726	0.0551374030972571\\
2.0966071953384	0.0446030864009579\\
2.15985263984993	0.0318215829913306\\
2.21891575983136	0.0210421612577836\\
2.27797887981279	0.0113253517264962\\
2.32696646689782	0.00403332697925585\\
2.39228324967787	-0.00466484966649539\\
2.45778879895274	-0.0122809809631708\\
2.52509814687654	-0.0190305358460741\\
2.59781889074711	-0.0251885592184991\\
2.65431087910301	-0.029221082055404\\
2.71470572836485	-0.03286189260829\\
2.79667978161753	-0.0367898242215112\\
2.8829965975809	-0.0397938762568639\\
2.96931341354427	-0.0417776843132316\\
3.06581505326621	-0.0429530019095488\\
3.15472043949996	-0.0432065373397279\\
3.26140690298045	-0.0426416328174248\\
3.38643770395599	-0.0410203653160774\\
3.52429526690554	-0.0383464001089244\\
3.71784295282373	-0.0335975508524342\\
4.47554880310426	-0.0137372264542996\\
4.70652223330217	-0.00908900509635835\\
4.93749566350007	-0.00539059197688019\\
5.16846909369798	-0.00260384039925121\\
5.42831420267062	-0.000431016481456936\\
5.71703099041801	0.00101177212217607\\
6.06349113571487	0.00176338617056437\\
6.52937477201933	0.00177169602552496\\
9.57339289197716	-7.90872404170528e-05\\
10	-6.27923251883544e-05\\
};
\addlegendentry{SFB}

\addplot [color=black, dashed]
  table[row sep=crcr]{%
0	1\\
0.00638301060643442	0.999557718223572\\
0.0126912168924367	0.998161508673201\\
0.018985708958116	0.995816403507469\\
0.0254564232875758	0.992414444113596\\
0.0322793291111356	0.987762698424648\\
0.0420241360416629	0.9798595808195\\
0.0588759034887048	0.964842437465521\\
0.109468984003882	0.918080772557861\\
0.225025448787214	0.811491858478876\\
0.29130398409238	0.751461643358182\\
0.35015868755775	0.699241851419545\\
0.398070916086603	0.657656412880074\\
0.453525546475106	0.610714842364217\\
0.508980176863609	0.565187145488958\\
0.565574895918418	0.520305709500203\\
0.622169614973227	0.477125624411213\\
0.678764334028036	0.435728648358944\\
0.735359053082844	0.396176412667419\\
0.791953772137653	0.358512255587305\\
0.848548491192462	0.322762967614953\\
0.905143210247271	0.288940447939719\\
0.961737929302078	0.257043271091561\\
1.01833264835689	0.227058164074512\\
1.06541722839523	0.203552059113317\\
1.11250180843357	0.1813345937781\\
1.15958638847191	0.160383317293119\\
1.20667096851025	0.140672338780609\\
1.25375554854858	0.122172758372686\\
1.30084012858693	0.104853071784532\\
1.34792470862526	0.0886795486244782\\
1.3950092886636	0.0736165854678976\\
1.44209386870194	0.0596270343901839\\
1.48493513783515	0.0477985687901157\\
1.52353309606323	0.0378509512551926\\
1.56213105429132	0.0285528460052014\\
1.6007290125194	0.01988201328599\\
1.65469132402067	0.00876863599659039\\
1.70078438384025	0.000165896544761068\\
1.74687744365983	-0.00765724078903141\\
1.79346587235917	-0.0148109973109953\\
1.84005430105852	-0.0212460138775032\\
1.89158476280141	-0.0275727744226408\\
1.94558624106608	-0.033365387687855\\
2.00151131331392	-0.0385212090429228\\
2.06128357352814	-0.0431527061830508\\
2.12105583374235	-0.0469477954600066\\
2.18373216626074	-0.0501060679508694\\
2.24640849877914	-0.05250052489456\\
2.32997694213699	-0.0546330975195204\\
2.40262519035646	-0.055620115046958\\
2.48888541932313	-0.0558915616053515\\
2.57514564828981	-0.0553335665749852\\
2.6788167113199	-0.0537665097054685\\
2.78911468642566	-0.0512586796964154\\
2.93013806802011	-0.0471504687196393\\
3.12628702356124	-0.0404520898140923\\
3.61899530513762	-0.0231382582981912\\
3.83834700226539	-0.016562587087634\\
4.03339820794692	-0.0116213448809752\\
4.22844941362846	-0.00757066983598165\\
4.42350061930999	-0.00437567045327292\\
4.64641628294603	-0.00167036497711948\\
4.86933194658206	0.000173856456106236\\
5.14797652612711	0.00152250949344257\\
5.48235002158117	0.00214552639711663\\
5.93207004839526	0.00200505642176729\\
8.30065475198119	-6.09405361799986e-05\\
10	-3.07893413502569e-05\\
};
\addlegendentry{$\text{OFB, sat} = 24$}

\addplot [color=black, dotted]
  table[row sep=crcr]{%
0	1\\
0.0149675997785756	0.999554914022685\\
0.0298457361472142	0.998224397098364\\
0.0448770696520793	0.995981021679892\\
0.0601958463212409	0.992785874741609\\
0.0808803178490685	0.987497690923682\\
0.105631683664743	0.98024256172955\\
0.131612090885657	0.971704227340053\\
0.158758835828447	0.961850539467985\\
0.18676151691969	0.950758704263047\\
0.216065480328229	0.938218679337607\\
0.246611402209552	0.924214219716108\\
0.281157667667905	0.907332668991716\\
0.30873012056529	0.893139402904351\\
0.352369031300595	0.869517808575321\\
0.397388942631942	0.843858568361247\\
0.445170855155375	0.815433341492891\\
0.492952767678808	0.786019063769208\\
0.559594056733664	0.743764811263413\\
0.648449108806805	0.68605534227213\\
0.841842812389981	0.559888493357583\\
0.917327775250547	0.511924889734185\\
0.992812738111112	0.465308733899978\\
1.04627764759745	0.433279775253185\\
1.09974255708378	0.402176711571373\\
1.15320746657012	0.37207648701264\\
1.20748243858041	0.342610985083006\\
1.26256747311466	0.313883506362341\\
1.31765250764891	0.286384827667595\\
1.37273754218316	0.260145641288954\\
1.4278225767174	0.23518589152755\\
1.48290761125165	0.211515811185778\\
1.5379926457859	0.189136904183069\\
1.59307768032015	0.168042874269503\\
1.64816271485439	0.14822050012331\\
1.70324774938864	0.129650457425793\\
1.74892387457538	0.115184558350174\\
1.79459999976211	0.101546348513965\\
1.84027612494885	0.0887174174185397\\
1.88595225013559	0.0766776861601688\\
1.93162837532232	0.0654056398408969\\
1.97730450050906	0.0548785451245255\\
2.0229806256958	0.0450726531374581\\
2.06865675088254	0.0359633883423598\\
2.12921361505207	0.0249172707642575\\
2.18579181741632	0.0156101651331131\\
2.24237001978056	0.00723105118642486\\
2.29894822214481	-0.000270083689638767\\
2.35552642450906	-0.00694321387326013\\
2.41210462687331	-0.0128379683101905\\
2.46868282923756	-0.0180033592994882\\
2.54429965985844	-0.0238515807100477\\
2.62009571794789	-0.0286048290153982\\
2.7010334804612	-0.032577264624889\\
2.78368514444912	-0.0355862203620241\\
2.86633680843705	-0.0376628134527106\\
2.94898847242498	-0.0389263144170666\\
3.05210186241616	-0.0395304995381469\\
3.15576504801192	-0.0392354064000493\\
3.27843367194545	-0.0379624853436038\\
3.42915816167312	-0.0353898473126755\\
3.61663988062166	-0.0312058589610515\\
4.44051127680965	-0.0115754197924911\\
4.66556470059191	-0.00762578367819522\\
4.89061812437417	-0.00451091282395666\\
5.14380322612922	-0.00193798838187575\\
5.42512000585705	-6.791908701409e-05\\
5.73456846355766	0.00105682275091468\\
6.12906378859397	0.00153674185004959\\
6.72900426922911	0.00124902051450348\\
8.47451478001637	2.2757634253523e-05\\
10	-4.40216865378318e-05\\
};
\addlegendentry{$\text{OFB, sat = } 4$}

\end{axis}

\begin{axis}[%
width=\figurewidth,
height=0.465\figureheight,
at={(0\figurewidth,0\figureheight)},
scale only axis,
xmin=0,
xmax=10,
xlabel style={font=\color{white!15!black}},
xlabel={Time, $t$ [\si{\second}]},
ymin=-0.95,
ymax=0.08,
ylabel style={font=\color{white!15!black}},
ylabel={$x_2(t)$},
axis background/.style={fill=white},
xlabel style={font=\footnotesize},ylabel style={font=\footnotesize},ticklabel style={font=\scriptsize},x tick label style={font=\scriptsize},y tick label style={font=\scriptsize},legend style={font=\scriptsize}
]
\addplot [color=black, forget plot]
  table[row sep=crcr]{%
0	0\\
0.0190086209298173	-0.03729916523654\\
0.0367363819427684	-0.0708066879405624\\
0.0606493011849629	-0.114090598966456\\
0.0831047969961443	-0.15277915936759\\
0.105560292807326	-0.189617961695983\\
0.128015788618507	-0.224651915623152\\
0.139243536524097	-0.241506057053892\\
0.157523732279078	-0.268019711761189\\
0.175803928034059	-0.293405641626192\\
0.19408412378904	-0.317687673221434\\
0.212364319544019	-0.340889502138996\\
0.230644515299	-0.363034682169355\\
0.248924711053981	-0.384146614934043\\
0.267204906808962	-0.404248539976688\\
0.287314314632789	-0.425222833950917\\
0.307423722456617	-0.445033241554381\\
0.327533130280445	-0.463709940858665\\
0.347642538104273	-0.481282818953179\\
0.3677519459281	-0.49778145973843\\
0.387861353751928	-0.51323513242434\\
0.407970761575756	-0.527672780725444\\
0.429537038275953	-0.542059845127206\\
0.451103314976148	-0.555346080540048\\
0.472669591676345	-0.567565863797025\\
0.494235868376542	-0.578753091343955\\
0.515802145076737	-0.588941168677472\\
0.537368421776934	-0.598163000660941\\
0.558934698477131	-0.606450982695096\\
0.581814200366589	-0.614258310157062\\
0.604693702256046	-0.621087885728741\\
0.627573204145504	-0.62697645369845\\
0.650452706034962	-0.63196007774811\\
0.673332207924419	-0.636074134511043\\
0.696211709813877	-0.639353308106367\\
0.719091211703335	-0.641831585614293\\
0.74320522717408	-0.643613367432872\\
0.767319242644827	-0.644580586248654\\
0.791433258115573	-0.644770496511939\\
0.815547273586319	-0.644219481039043\\
0.839661289057064	-0.642963050766845\\
0.863775304527811	-0.641035845495326\\
0.887889319998557	-0.638471635570603\\
0.913301097684426	-0.635116308519775\\
0.938712875370294	-0.631127568458417\\
0.964124653056164	-0.626541825568856\\
0.989536430742033	-0.62139443745578\\
1.0149482084279	-0.615719716805604\\
1.04035998611377	-0.609550939968621\\
1.06577176379964	-0.602920356406676\\
1.09269747268888	-0.595425593722245\\
1.11962318157811	-0.587483336478122\\
1.17347459935659	-0.570393787565765\\
1.22732601713506	-0.551914364115015\\
1.28302891166586	-0.531596161353969\\
1.34058328294898	-0.509577897161336\\
1.39813765423211	-0.486766551972005\\
1.4844692111568	-0.451557369302126\\
1.76242675867043	-0.336972423040251\\
1.83403685224214	-0.308527229685046\\
1.90564694581385	-0.28093208173566\\
1.97725703938556	-0.254337009881839\\
2.04886713295726	-0.2288624772165\\
2.0966071953384	-0.212549264368691\\
2.15985263984993	-0.19180594054977\\
2.21891575983136	-0.173358669178558\\
2.27797887981279	-0.155826583015454\\
2.32696646689782	-0.141989383737178\\
2.37595405398286	-0.128794683011153\\
2.42503602431531	-0.11621925447812\\
2.47416518627146	-0.104275325976575\\
2.52509814687654	-0.0925669088117189\\
2.57963870477947	-0.0807801191505302\\
2.6341792626824	-0.0697563244616575\\
2.69457411194424	-0.0584191717645037\\
2.75496896120608	-0.0479707612347919\\
2.81825898560838	-0.0379431002760917\\
2.8829965975809	-0.0286240536006428\\
2.94773420955343	-0.0202118615310329\\
3.01247182152596	-0.0126624825761734\\
3.08359613051296	-0.00530883726785802\\
3.15472043949996	0.00111977948444952\\
3.22584474848695	0.00668544611587585\\
3.297062934182	0.011455927904759\\
3.36856275000119	0.0154994132351192\\
3.44430686090904	0.0190355829645839\\
3.52429526690554	0.0220118230489046\\
3.60888542392316	0.0243993238303961\\
3.69500949794782	0.0261192535209034\\
3.78634331745148	0.0272593744582608\\
3.90298343956517	0.0278477532632042\\
4.02333283328004	0.0276216542390451\\
4.17551046400639	0.0264106080614965\\
4.36206800044872	0.0239549336451024\\
4.61990719697795	0.0196232035373303\\
5.16846909369798	0.0102226883999066\\
5.42831420267062	0.00662513047161895\\
5.68815931164327	0.00379570820565434\\
5.94800442061591	0.00171775436186294\\
6.23672430998176	0.000181713137521911\\
6.58907566185849	-0.000828328766854014\\
7.03086948893093	-0.00120235969461469\\
7.73061061902038	-0.00087167579597569\\
9.33255988446066	-1.62985172806884e-05\\
10	4.93971132176085e-05\\
};
\addplot [color=black, dashed, forget plot]
  table[row sep=crcr]{%
0	0\\
0.000326366756565832	-0.00209324181574644\\
0.000985634137920854	-0.0160967487684349\\
0.0311541067576062	-0.730988838746079\\
0.0342317153366558	-0.779808731445183\\
0.037316172924136	-0.816603967441372\\
0.0403404270207979	-0.843750253940449\\
0.0435050636882757	-0.865034480432303\\
0.0467774517845783	-0.881329424378587\\
0.0499941531923831	-0.893158789775033\\
0.0531306611177538	-0.901738010400027\\
0.0563436184042416	-0.908286556353696\\
0.0595073312617398	-0.913092127264546\\
0.0628920453362021	-0.916904218985453\\
0.0667333255674532	-0.920025358002063\\
0.070910903870308	-0.92239647038422\\
0.075963511129391	-0.924306458661052\\
0.0820110028097911	-0.925715109951554\\
0.089933822342747	-0.9267139694558\\
0.101113515378366	-0.927258812651866\\
0.116318783968252	-0.927180655144069\\
0.135855391501632	-0.926267870726386\\
0.156608826669469	-0.924515007698885\\
0.17888284211203	-0.921815000533284\\
0.201495293995755	-0.918253863843292\\
0.225025448787214	-0.913715997836055\\
0.24685336248932	-0.90878663689365\\
0.271261464661061	-0.902498570425459\\
0.29130398409238	-0.896754959716574\\
0.320731335825066	-0.887431136829296\\
0.35015868755775	-0.877118007178566\\
0.379586039290434	-0.865891718692637\\
0.41655579288277	-0.850606241024472\\
0.453525546475106	-0.834133762648035\\
0.490495300067442	-0.816606165392093\\
0.537277536391015	-0.793108476192398\\
0.593872255445824	-0.76303024301739\\
0.650466974500631	-0.731508799789998\\
0.70706169355544	-0.69889971116479\\
0.791953772137653	-0.648645193401093\\
1.04187493837606	-0.499215019161049\\
1.11250180843357	-0.458331102142377\\
1.18312867849108	-0.418603549335355\\
1.23021325852942	-0.392873118960454\\
1.27729783856775	-0.36781212348699\\
1.32438241860609	-0.3434678190083\\
1.37146699864443	-0.319879964086807\\
1.41855157868277	-0.29708134046413\\
1.46563615872111	-0.275098255973802\\
1.52353309606323	-0.249214514171763\\
1.56213105429132	-0.232674147881477\\
1.6007290125194	-0.216712277786323\\
1.65469132402067	-0.195371453403892\\
1.70078438384025	-0.178043767619462\\
1.74687744365983	-0.161543390693277\\
1.79346587235917	-0.145700192585922\\
1.84005430105852	-0.130686746005207\\
1.89158476280141	-0.115032666307574\\
1.94558624106608	-0.0996801535298353\\
2.00151131331392	-0.0848882776379707\\
2.04135948679007	-0.0750184819431308\\
2.10113174700428	-0.0612275970203466\\
2.14194794458181	-0.0524885844964853\\
2.18373216626074	-0.0440939511173504\\
2.24640849877914	-0.0325110864378875\\
2.30908483129753	-0.022085968893597\\
2.36812109876979	-0.0132702869882202\\
2.4198772361498	-0.00630112723721865\\
2.4716333735298	-4.67040106855166e-06\\
2.5233895109098	0.00565420709264153\\
2.59239769408314	0.012267870573373\\
2.66152496845905	0.017900608518584\\
2.73069193990248	0.0226228447052712\\
2.80858893526673	0.0269441355748867\\
2.88805381053331	0.0303713318218524\\
2.97222232550691	0.0330408918586294\\
3.05909875922072	0.0348868954028987\\
3.14868311167475	0.0359542501594809\\
3.24214765885615	0.0362990049763852\\
3.36196043633486	0.0358001533014907\\
3.4872345966091	0.0343966406816492\\
3.64567177425604	0.0317044761928358\\
3.83834700226539	0.0275954752885692\\
4.5070939931735	0.0126410448321099\\
4.73000965680954	0.00868378894831068\\
4.95292532044558	0.00547748908953771\\
5.17584098408162	0.00300665436652459\\
5.42662110567216	0.001018010411304\\
5.7053992514512	-0.000376468880460834\\
6.04808950384399	-0.00121418995387224\\
6.48508671642083	-0.00140653799349089\\
7.32376361657528	-0.000766423258435012\\
8.56905879809253	-3.97167562855572e-05\\
10	4.63777710564273e-05\\
};
\addplot [color=black, dotted, forget plot]
  table[row sep=crcr]{%
0	0\\
0.000915415483785154	-0.00346265029242865\\
0.0552331414399383	-0.219306253874649\\
0.059497349230524	-0.231889530737046\\
0.0643868288655387	-0.243814529298369\\
0.07014999676875	-0.255688449502166\\
0.0770206941938021	-0.268002743012419\\
0.0860768459307124	-0.282551899545965\\
0.0983216381627052	-0.30067497024074\\
0.114646149748729	-0.323395406579237\\
0.133562928064681	-0.348364388860585\\
0.153133383065448	-0.372884297819303\\
0.173052693355556	-0.396541900577528\\
0.191587028434435	-0.417411912560704\\
0.209858143774763	-0.436934843804634\\
0.229623468568651	-0.456909352081418\\
0.250858385619777	-0.477078441958112\\
0.268131518348953	-0.492524931118719\\
0.28767074232738	-0.508988115985838\\
0.30873012056529	-0.525564679986379\\
0.323276424143726	-0.536326065390965\\
0.337822727722161	-0.546538628819567\\
0.352369031300595	-0.556214201322756\\
0.36691533487903	-0.565364495100869\\
0.381461638457465	-0.574001102017556\\
0.397388942631942	-0.58288206686005\\
0.413316246806419	-0.59117581466049\\
0.429243550980898	-0.598897054350115\\
0.445170855155375	-0.60606031426382\\
0.461098159329852	-0.612679940523842\\
0.477025463504331	-0.618770095535991\\
0.492952767678808	-0.624344756613752\\
0.515166530697094	-0.631284901719525\\
0.53738029371538	-0.637285920585237\\
0.559594056733664	-0.642383904093496\\
0.581807819751949	-0.646614225355512\\
0.604021582770235	-0.650011536696024\\
0.626235345788521	-0.652609767431871\\
0.648449108806805	-0.654442122455778\\
0.67248225870628	-0.655599578073193\\
0.696515408605753	-0.655938747042208\\
0.720548558505229	-0.65549881405083\\
0.744581708404702	-0.654317974387649\\
0.768614858304177	-0.652433437407661\\
0.792648008203651	-0.649881430960933\\
0.816681158103126	-0.646697206727255\\
0.841842812389981	-0.642723315781252\\
0.867004466676837	-0.638132218915407\\
0.892166120963692	-0.632960979654685\\
0.917327775250547	-0.627245525332448\\
0.942489429537403	-0.621020658844227\\
0.967651083824258	-0.614320071237847\\
0.992812738111112	-0.607176355053012\\
1.01954519285428	-0.599136400329977\\
1.07301010234062	-0.581807106577175\\
1.12647501182695	-0.563039525961402\\
1.17993992131329	-0.54308862302257\\
1.23502495584754	-0.521543563221133\\
1.29010999038178	-0.499227283654362\\
1.37273754218316	-0.46477140501994\\
1.64816271485439	-0.348416196212625\\
1.72608581198201	-0.316690840895392\\
1.79459999976211	-0.289655807576713\\
1.86311418754222	-0.263569505108858\\
1.93162837532232	-0.23854235340897\\
2.00014256310243	-0.214660658558339\\
2.06865675088254	-0.191989012011614\\
2.12921361505207	-0.172994434117596\\
2.18579181741632	-0.156151793176951\\
2.24237001978056	-0.140191468473121\\
2.29894822214481	-0.125115721374804\\
2.35552642450906	-0.110921051888278\\
2.41210462687331	-0.0975988551308564\\
2.46868282923756	-0.0851360373075476\\
2.52535064533608	-0.0734978440644447\\
2.58219768890317	-0.0626512940210553\\
2.63904473247025	-0.0526109242272792\\
2.7010334804612	-0.042550621952925\\
2.76302222845214	-0.0333813959982479\\
2.82501097644309	-0.0250644924685499\\
2.88699972443403	-0.0175594691383747\\
2.94898847242498	-0.0108246989499126\\
3.0177307324191	-0.00420554733196887\\
3.08658295153414	0.00158529899641557\\
3.15576504801192	0.00661716460010986\\
3.22558712652866	0.0109555155512577\\
3.29604918708437	0.0146398580303355\\
3.37141143194821	0.017875902053337\\
3.44840707158142	0.0205012096543999\\
3.53185016676842	0.0226516111565012\\
3.61663988062166	0.0241829382746648\\
3.72767684293666	0.0253320643872055\\
3.84375494854515	0.0256699307988875\\
3.96371212991288	0.0252800460936253\\
4.11390088372565	0.0239956856655965\\
4.29985288694573	0.0215813247317165\\
4.60930134464635	0.0166171361173468\\
5.03127651423809	0.00993525718103605\\
5.31259329396592	0.00628426420757755\\
5.59391007369375	0.00347256467724399\\
5.87522685342158	0.00147432622956067\\
6.18701845756956	7.07755368054563e-05\\
6.54155555434972	-0.000740211093170018\\
7.02600646781781	-0.00100243300217961\\
7.98858874610414	-0.000498253534090054\\
9.4134485420468	7.35279192198846e-06\\
10	3.86793456019774e-05\\
};
\end{axis}
\end{tikzpicture}%
    \selectlanguage{greek}
    \caption{Οι $x$-τροχιές του κλειστού βρόγχου υπό ανάδραση εξόδου και καταστάσεων, για διαφορετικά σημεία κορεσμού.}
    \label{fig:hgo_sfb_ofb_sat_comparison}
\end{figure}

