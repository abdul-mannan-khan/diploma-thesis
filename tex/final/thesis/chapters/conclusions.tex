\chapter{Συμπεράσματα και ανοιχτά προβλήματα}
\label{chap:conclusions}

Για την κλάση των ΠΕΠΕ, υψηλού σχετικού βαθμού, αβέβαιων μη-γραμμικών συστημάτων σε κανονική μορφή, με εσωτερικές δυναμικές με ευστάθειας φραγμένης-εισόδου-φραγμένων-καταστάσεων, και υπό παρουσία εξωτερικών φραγμένων διαταραχών, προτείνουμε έναν χαμηλής πολυπλοκότητας νόμο ελέγχου που λειτουργεί υπό ανάδραση εξόδου, χωρίς να κάνει χρήση κάποιου εσωτερικού μοντέλου είτε δομών εκτίμησης, χρησιμοποιώντας τη μεθοδολογία ελέγχου ΠΑΕ και παρατηρητές υψηλού κέρδους. Ο ελεγκτής αυτός εγγυάται ότι
\begin{enumerate*}[label=(\alph*)]
    \item το σφάλμα παρακολούθησης εξόδου θα συγκλίνει σε μία προδιαγεγραμμένη τελική ζώνη με προκαθορισμένο ελάχιστον ρυθμό σύγκλισης, και 
    \item όλα τα σήματα του κλειστού βρόχου θα είναι φραγμένα.
\end{enumerate*}
Για να αποφύγουμε τη μετάδοση του φαινομένου κορύφωσης στο σύστημα υπό έλεγχο, η είσοδος ελέγχου υπόκειται σε κορεσμό. Όμως το επίπεδο κορεσμού επιλέγεται κατάλληλα ώστε να αφήνει τα χαρακτηριστικά απόκρισης του κλειστού βρόχου ανεπηρέαστα. Μελέτες προσομοίωσης επιβεβαιώνουν και σε πρακτικό επίπεδο τα θεωρητικά αποτελέσματα.

Κλείνοντας, θα θέλαμε να αναφέρουμε μερικά ανοιχτά ζητήματα τα οποία θα μπορούσαν να βελτιωθούν/λυθούν σε μελλοντικές εργασίες.
\begin{itemize}
    \item Το προτεινόμενο σχήμα ελέγχου, όπως εξηγείται και στην \cref{remark:overshoot}, δεν μπορεί να προδιαγράψει το ποσοστό υπερύψωησης στο σφάλμα παρακολούθησης εξόδου. Ενδεχομένως, με μια διαφορετική προσέγγιση αυτό να είναι εφικτό, καθώς το σφάλμα αυτό είναι μετρήσιμο εξ υποθέσεως.
    \item Το προτεινόμενο σχήμα επιβάλει τα χαρακτηριστικά απόκρισης σε ένα γενικευμένο σφάλμα, όπως αναφέρεται και στην \cref{remark:manifold}. Αυτές οι προδιαγραφές μεταφέρονται στις επιμέρους καταστάσεις, αλλά όχι πλήρως: το \cref{lemma} εγγυάται πως θα υπάρχουν κάποιες συναρτήσεις επίδοσης οι οποίες θα φράσσουν τα σφάλματα παρακολούθησης εξόδου και τις παραγώγους τους, αλλά με άγνωστες αρχικές συνθήκες. Συνεπώς, δεν είναι εφικτό στο τρέχον πλαίσιο να γνωρίζουμε επακριβώς το χρόνο στον οποίο το σφάλμα εξόδου θα είναι φραγμένο κάτω από μία επιθυμητή τιμή, παρά μόνο έναν γενικό ρυθμό σύγκλισης αυτού.
    \item Αν κάποιος προσπαθήσει να εφαρμόσει αυτόν τον ελεγκτή σε κάποιο πραγματικό σενάριο, είναι πολύ πιθανό να καταλήξει σε μια διαδικασία \textlatin{trial and error} προκειμένου να βρει κατάλληλες σταθερές $u_i^\star$, $i = 1,\ldots,m$ και $\mu$, ώστε να πετύχει τους επιθυμητούς στόχους ελέγχου. Μία πιθανή ερευνητική κατεύθυνση λοιπόν, είναι η εξής: δοσμένης κάποιας πληροφορίας για το σύστημα να μπορέσει ο σχεδιαστής να αποκτήσει κάποια πληροφορία για την επιλογή των μεγεθών αυτών. Επίσης ένα άλλο ανοιχτό θέμα είναι και η αντίστροφη διαδικασία: δεδομένου ενός περιορισμού στην είσοδο που διαθέτει ένα φυσικό σύστημα, ποια είναι τα χαρακτηριστικά απόκρισης που μπορούν να του επιβληθούν.
\end{itemize}