\chapter{Εισαγωγή}
Ο έλεγχος μη-γραμμικών συστημάτων παρουσία αβεβαιοτήτων είναι ερευνητικό αντικείμενο που, λόγω της δυσκολίας και της πολυπλοκότητας του, αποτελεί ένα ιδιαίτερα ενδιαφέρον θέμα για την ακαδημαϊκή κοινότητα, και ως εκ τούτου πολλές εργασίες έχουν δημοσιευτεί σχετικά με αυτό. H πλειονότητα των προσπαθειών αυτών καταφέρνει να εγγυηθεί ότι το σφάλμα εξόδου θα καταλήξει σε μία περιοχή του μηδενός, το μέγεθος όμως της οποίας εξαρτάται από τις παραμέτρους του σχήματος ελέγχου αλλά και από άγνωστες φραγμένες σταθερές που εξαρτώνται από τις αβεβαιότητες των μη-γραμμικοτήτων, και επομένως δεν μπορεί να καθοριστεί αναλυτικά. Επιπλέον, δεν υπάρχει κάποια διαδικασία η οποία να μπορεί με συστηματικό τρόπο να εγγυηθεί χαρακτηριστικά απόκρισης του σφάλματος κατά το μεταβατικό του, όπως χρόνος σύγκλισης ή μέγιστη υπερύψωση. Ο τρόπος που επιτυγχάνονται τέτοιου είδους στόχοι ελέγχου είναι μέσω μιας επαναληπτικής διαδικασίας, κατά την οποία ρυθμίζουμε τις ελεύθερες παραμέτρους του ελεγκτή μέχρι να πετύχουμε την επιθυμητή συμπεριφορά. Αυτή η διαδικασία είναι χρονοβόρα και προφανώς δεν προσφέρει εγγυήσεις, καθώς μία μικρή αλλαγή στις παραμέτρους του συστήματος υπό έλεγχο ή η οποιαδήποτε διαταραχή μπορεί να αναιρέσει όλη τη προσπάθεια που είχαμε καταβάλει. Συνεπώς, υπάρχει μια πραγματική ανάγκη για ένα σχήμα ελέγχου το οποίο θα μπορεί να εγγυηθεί, με εύρωστο τρόπο, τα επιθυμητά χαρακτηριστικά απόκρισης για τα σφάλματα εξόδου. Εξ'όσων γνωρίζει ο συγγραφέας υπάρχουν δύο μεγάλες οικογένειες ελεγκτών στη διεθνή βιβλιογραφία που να μπορούν να επιβάλουν χαρακτηριστικά απόκρισης επιλεγμένα εκ των προτέρων στο σφάλμα εξόδου, οι οποίες θα αποτελέσουν το αντικείμενο συζήτησης των επόμενων παραγράφων.

Η μία από αυτές τις μεθοδολογίες είναι γνωστή ως \textlatin{\emph{funnel control}} και παρουσιάστηκε για πρώτη φορά στην εργασία~\cite{ilchmann2002tracking} και αφορούσε μη-γραμμικά συστήματα σχετικής τάξης ένα. Αργότερα, στην εργασία~\cite{ilchmann2007tracking}, η μεθοδολογία αυτή επεκτάθηκε θεωρητικά ώστε να μπορεί να εφαρμοστεί και σε συστήματα οποιουδήποτε σχετικού βαθμού χρησιμοποιώντας την τεχνική \textlatin{backstepping} σε συνδυασμό με ένα φίλτρο/αντισταθμιστή. Όμως αυτή η \textlatin{backstepping} διαδικασία παραμένει θεωρητική, καθώς οποιαδήποτε πρακτική εφαρμογή της σε συστήματα ανώτερης τάξης, πέραν της περιπλοκότητας του νόμου ελέγχου, περιλαμβάνει μεγάλες δυνάμεις μίας συνάρτησης κέρδους η οποία ούτως ή άλλως παράγει μεγάλες τιμές~\cite[Παρατήρηση 2.3]{Berger2018},~\cite{Hackl2012}. Στην εργασία~\cite{Hackl2013} προτάθηκε ένας αναλογικός-διαφορικός ελεγκτής με μη-γραμμικό κέρδος για συστήματα σχετικού βαθμού δύο. Επεκτάσεις αυτού του ελεγκτή για γνωστή, αλλά αυθαίρετη σχετική τάξη, παρουσιάζονται στις εργασίες~\cite{Liberzon2013}, οπού προτάθηκε o \textlatin{\emph{bang-bang funnel controller}}, και~\cite{Berger2018}, όπου προτείνεται ένας ελεγκτής που λειτουργεί μέσω μιας αναδρομικής διαδικασίας. Η εργασία~\cite{Liberzon2013} αφορά συστήματα μίας-εισόδου-μίας-εξόδου, και οι συνθήκες που παρουσιάζονται για την πραγμάτωση του ελεγκτή είναι ιδιαίτερα περίπλοκες. Η εργασία~\cite{Berger2018} αφορά πολλαπλών-εισόδων-πολλαπλών-εξόδων συστήματα, με αυστηρό σχετικό βαθμό και εσωτερική δυναμική ευσταθή από την είσοδο στην κατάσταση. Μια διαφορετική προσέγγιση της επέκτασης του \textlatin{funnel control} για συστήματα με ανώτερη σχετική τάξη παρουσιάζεται στην εργασία~\cite{chowdhury2017funnel}, όπου  κατασκευάζεται μια εικονική έξοδος και ο σκοπός του ελεγκτή είναι να εφαρμόσει τα ζητούμενα χαρακτηριστικά απόκρισης σε αυτή, το οποίο μπορεί να μεταφραστεί και σε χαρακτηριστικά απόκρισης για το σφάλμα εξόδου. Τα συστήματα υπό συζήτηση σε αυτή την εργασία είναι μίας-εισόδου-μίας-εξόδου, με τις καταστάσεις της εσωτερικής δυναμικής φραγμένες για φραγμένη είσοδο, αλλά δεν μπορούν να είναι εντελώς αβέβαια, καθώς απαιτείται μερική γνώση των μη-γραμμικοτήτων. Αν επιθυμεί κανείς περισσότερες λεπτομέρειες πάνω στο συγκεκριμένο θέμα, τον προτρέπουμε στη βιβλιογραφική αναφορά~\cite{ilchmann2008high}.

Η άλλη μεγάλη οικογένεια ελεγκτών που επιλύει το πρόβλημα υπό συζήτηση είναι οι ελεγκτές \emph{προδιαγεγραμμένης απόκρισης εξόδου (ΠΑΕ)}, στην οποία υπάγεται και η παρούσα εργασία. Γνωστή ως \textlatin{\emph{prescribed performance control (PPC)}} στη διεθνή βιβλιογραφία, παρουσιάστηκε πρώτη φορά στην εργασία~\cite{bechlioulis2008robust}, που αφορούσε γραμμικοποιήσιμα μέσω ανάδρασης πολλαπλών-εισόδων-πολλαπλών-εξόδων συστήματα, με τετριμμένες εσωτερικές δυναμικές. Ο προτεινόμενος ελεγκτής είναι ιδιαίτερα πολύπλοκος στην υλοποίηση, καθώς απαιτεί δομές νευρωνικών δικτύων για την αναγνώριση του συστήματος υπό έλεγχο. Αργότερα, στην εργασία~\cite{Theodorakopoulos2016}, η οποία αφορούσε την ίδια κλάση συστημάτων, προτάθηκε ένας απλούστερος ελεγκτής, χωρίς να απαιτεί δομή αναγνώρισης, που επιλύει το ίδιο πρόβλημα. Επιπλέον, έλεγχος ΠΑΕ έχει εφαρμοστεί επιτυχώς και σε κασκοδικά συστήματα με μερική ανάδραση καταστάσεων~\cite{bechlioulis2011robust}, σε συστήματα αυστηρής~\cite{bechlioulis2009adaptive} και γνήσιας ανάδρασης~\cite{bechlioulis2014low}. Οι πρώτες εργασίες~\cite{bechlioulis2008robust, bechlioulis2009adaptive, bechlioulis2010prescribed} κινούταν στα πλαίσια του εύρωστου προσαρμοστικού ελέγχου, ενώ οι μετέπειτα απόπειρες~\cite{bechlioulis2011robust, bechlioulis2014low, Theodorakopoulos2016} ακολουθούν λογική σχεδίασης χωρίς εσωτερικό μοντέλο, μειώνοντας σημαντικά την πολυπλοκότητα της υλοποίησης.


Και οι δύο οικογένειες ελεγκτών που αναφέρθηκαν ανήκουν στην μεγαλύτερη οικογένεια ελεγκτών υψηλού κέρδους (\textlatin{high gain control}). Σε αυτή την κατηγορία ανήκει και η εργασία~\cite{bullinger2005adaptive}, στην οποία επιτυγχάνεται ασυμπτωτική παρακολούθηση τροχιάς με προδιαγεγραμμένη ακρίβεια αλλά ο προτεινόμενος ελεγκτής δεν είναι ικανός να προσφέρει εγγυήσεις για το μεταβατικό φαινόμενο. Η αρχή λειτουργίας αυτών των ελεγκτών, όπως άλλωστε μαρτυράει και το όνομα, είναι πως, αρκούντως υψηλού κέρδους, μπορούμε να πετύχουμε επαρκή απόσβεση των διαταραχών\footnote{Θεωρούμε τις μη-γραμμικότητες του συστήματος για τις οποίες δεν έχουμε γνώση και προφανώς δεν αξιοποιούνται από τον ελεγκτή ως διαταραχές του συστήματος.} ώστε να επιτευχθεί ο στόχος ελέγχου. Η αρχή λειτουργίας του \textlatin{funnel control} έγκειται στη χρήση ενός χρονικά μεταβαλλόμενου μη-γραμμικού κέρδους, 
\[
    u = - k(t) e(t),
\]
το οποίο μεταβάλλεται ανάλογα με το πόσο κοντά βρίσκεται το σφάλμα στην περιβάλλουσα συνάρτηση επίδοσης ή αλλιώς το πόσο κοντά είναι το σφάλμα ώστε να ξεφύγει από τις επιθυμητές προδιαγραφές απόκρισης: όσο πιο κοντά είναι το σφάλμα στο να παραβιάσει τα προδιαγεγραμμένα κριτήρια, τόσο μεγαλύτερη διορθωτική δράση απαιτείται από τον ελεγκτή. Από την άλλη, οι ελεγκτές ΠΑΕ χρησιμοποιούν ένα μετασχηματισμό $T(\cdot)$, μεταφέροντας τα σφάλματα παρακολούθησης σε ένα διαφορετικό σύστημα συντεταγμένων, και αυτά τα μετασχηματισμένα σφάλματα είναι που αναδρόνται στο σύστημα μέσω ενός νόμος ελέγχου της μορφής:
\[
    u = - k T(e(t)).
\]
Αποδεικνύοντας το φραγμένο των μετασχηματισμένων αυτών σφαλμάτων, μπορούμε να εγγυηθούμε ότι τα σφάλματα στις αρχικές συντεταγμένες θα τηρούν τις προδιαγραφές απόκρισης. 6Όμως, περισσότερες τεχνικές λεπτομέρειες δεν είναι της παρούσης, αλλά θα δοθούν σε μετέπειτα κεφάλαιο που θα αναλύει εκτενώς τη μαθηματική θεώρηση και αρχή λειτουργίας των ελεγκτών ΠΑΕ.

Ένα ακόμα αντικείμενο με έντονο ερευνητικό ενδιαφέρον είναι ο έλεγχος μη-γραμμικών συστημάτων όταν δεν είναι όλες οι καταστάσεις διαθέσιμες προς μέτρηση, παρά μόνο η έξοδος. Η σχεδίαση ελεγκτών ανάδρασης εξόδου συχνά αντιμετωπίζεται με τη χρήση \emph{παρατηρητών υψηλού κέρδους}~\cite{khalil2008high,Khalil2013}. Στη ξενόγλωσση βιβλιογραφία αναφέρονται ως \textlatin{\emph{high gain observers}} και χρησιμοποιούνται για την ανακατασκευή των μη-διαθέσιμων καταστάσεων από μετρήσεις της εξόδου, οι οποίες μετέπειτα χρησιμοποιούνται σε έναν ελεγκτή που έχει σχεδιαστεί υπό ανάδραση καταστάσεων. Με τη χρήση αυτής της μεθοδολογίας, που περιγράφεται αναλυτικά στις~\cite{khalil1996noninear,atassi2001separation,atassi1999separation}, η σχεδίαση ελέγχου είναι αδιάφορη της σχεδίασης του παρατηρητή και το αντίστροφο, καθώς αποδεικνύεται η αρχή του διαχωρισμού μεταξύ των δύο. Όσον αφορά την απόκριση του κλειστού βρόγχου υπό ανάδραση εξόδου, αποδεικνύεται στις εργασίες~\cite{atassi2001separation,atassi1999separation} πως, καθώς αυξάνουν τα κέρδη του παρατηρητή οι τροχιές του συστήματος συγκλίνουν στις ιδανικές τροχιές που θα υπήρχαν υπό πλήρη ανάδραση καταστάσεων. Επιπλέον, έχοντας γνώση των μη-γραμμικοτήτων του συστήματος, μπορούμε να επιτύχουμε ασυμπτωτικά τέλεια ανακατασκευή καταστάσεων.

Όμως παρόλο που και το \textlatin{funnel control} και ο έλεγχος ΠΑΕ είναι πεδία τα οποία έχουν ωριμάσει ερευνητικά, η βιβλιογραφία εμφανίζει κενά στο κομμάτι της ανάδρασης εξόδου. Στην οικογένεια του \textlatin{funnel control} έχουν υπάρξει προσπάθειες, όπως οι εργασίες~\cite{ilchmann2007tracking} και~\cite{chowdhury2017funnel}. Όπως αναφέρθηκε και προηγουμένως, το σχήμα ελέγχου της εργασίας~\cite{ilchmann2007tracking} είναι εντελώς μη-πρακτικό για οποιοδήποτε ρεαλιστικό σενάριο, ενώ η εργασία~\cite{chowdhury2017funnel} παρόλο που προτείνει μια πιο εφικτή λύση, αντιμετωπίζει μόνο συστήματα MEME και απαιτεί μερική γνώση των μη-γραμμικοτήτων του συστήματος. Από την άλλη, ελεγκτές ΠΑΕ με ανάδραση εξόδου έχουν προταθεί στις εργασίες~\cite{kostarigka2009approximate, kostarigka2012adaptive}, όπου υπάρχουν δομικοί περιορισμοί. Επιπροσθέτως, γίνεται χρήση δομών εκτίμησης, αυξάνοντας έτσι την πολυπλοκότητα του ελεγκτή. Πρόσφατα όμως, στην εργασία~\cite{bechlioulis2013output} προτάθηκε ένα σχήμα ελέγχου το οποίο, χωρίς χρήση κάποιου εσωτερικού μοντέλου και χωρίς καμία γνώση των μη-γραμμικοτήτων, είναι ικανό να πετύχει σταθεροποίηση ενός ΜΕΜΕ μη-γραμμικού συστήματος σε κανονική μορφή.

H παρούσα εργασία, σε μία συνέχιση της~\cite{bechlioulis2013output}, προσπαθεί να γεφυρώσει τον έλεγχο ΠΑΕ με την ανάδραση εξόδου προτείνοντας ένα σχήμα ελέγχου το οποίο δεν απαιτεί καμία μη-δομική γνώση για το σύστημα και καλύπτει μια ευρεία κλάση μη-γραμμικών συστημάτων πολλαπλών-εισόδων-πολλαπλών εξόδων με μη-τετριμμένες εσωτερικές δυναμικές και μπορεί να πετύχει το στόχο ελέγχου παρουσία διαταραχών χρησιμοποιώντας μόνο μετρήσεις της εξόδου, καθιστώντας την έτσι μία πλήρη λύση σε σύγκριση με τη βιβλιογραφία. Η διαδικασία σχεδίασης που ακολουθούμε είναι εμφανώς επηρεασμένη από τις εργασίες~\cite{atassi1999separation,atassi2001separation,khalil1996noninear} και αποτελείται από τρία στάδια. Αρχικά, σχεδιάζεται ένας ελεγκτής καταστάσεων που να είναι ικανός να επιβάλει τα επιθυμητά χαρακτηριστικά απόκρισης στο σύστημα υπό έλεγχο, κάνοντας προσωρινά την υπόθεση ότι το διάνυσμα καταστάσεων είναι διαθέσιμο προς μέτρηση. Έπειτα, η είσοδος ελέγχου υπόκειται κορεσμό, με το επίπεδο αυτού να βρίσκεται εκτός της περιοχής λειτουργίας του ελεγκτή υπό ανάδραση καταστάσεων. Τέλος, μετατρέπουμε τον ελεγκτή ανάδρασης καταστάσεων σε ελεγκτή ανάδρασης εξόδου, αντικαθιστώντας τις καταστάσεις στον νόμο ελέγχου με εκτιμήσεις αυτών τις οποίες μας παρέχει ένας παρατηρητής υψηλού κέρδους, που διαφορίζει το μετρήσιμο σφάλμα παρακολούθησης εξόδου. Με τη χρήση κορεσμού στην είσοδο αποτρέπουμε το φαινόμενο κορύφωσης από το να μεταφερθεί στο σύστημα υπό έλεγχο. 

Η δομή της υπόλοιπης εργασίας έχει ως εξής:
\begin{itemize}
    \item Στο \cref{chap:ppc} παρουσιάζεται η αρχή λειτουργίας του ελέγχου ΠΑΕ καθώς και η μαθηματική θεώρησή του, απαραίτητο υπόβαθρο για την κατανόηση του Κεφαλαίου 4.
    \item Στο \cref{chap:hgo} παρουσιάζεται μία εισαγωγή στους παρατηρητές υψηλού κέρδους καθώς και ένα παράδειγμα που επιδεικνύει την αρχή λειτουργίας τους.
    \item Στο \cref{chap:ofppc} παρουσιάζεται ο κύριος κορμός αυτής της εργασίας, ορίζοντας σαφώς το πρόβλημα προς αντιμετώπιση, παρουσιάζοντας το προτεινόμενο σχήμα έλεγχου και αποδεικνύοντας την ορθή λειτουργία του.
    \item Στο \cref{chap:sims} έχουμε κάποιες μελέτες προσομοίωσης, όπου επιδεικνύεται και πρακτικά το γεγονός ότι ο προτεινόμενος ελεγκτής λειτουργεί όπως εγγυάται η θεωρητική ανάλυση.
    \item Στο \cref{chap:conclusions} ανακεφαλαιώνουμε τη δουλειά και παρουσιάζουμε πιθανές μελλοντικές βελτιώσεις και ανοιχτά προβλήματα.
    \item Τέλος, στο \nameref{chap:appendix} παρουσιάζονται πολύ συνοπτικά διάφορα μαθηματικά εργαλεία και ορισμούς που χρησιμοποιήθηκαν κατά τη διάρκεια αυτής της εργασίας.
\end{itemize}

